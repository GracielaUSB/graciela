\documentclass{beamer}

\mode<presentation>
{
  \usetheme{Warsaw}
  \usecolortheme{default}
  \usefonttheme{default}
  \setbeamertemplate{navigation symbols}{}
  \setbeamertemplate{caption}[numbered]
}

\usepackage[spanish]{babel}
\usepackage[utf8x]{inputenc}
\usepackage[T1]{fontenc}
\usepackage{lmodern}
\usepackage{syntax}

\setlength{\grammarparsep}{5pt plus 1pt minus 1pt}
\setlength{\grammarindent}{7em}

\title[Graciela]{[Gg]raciela - Segunda entrega}
\author[Ackerman - Spaggiari]{Moisés Ackerman \and Carlos Spaggiari}
\institute[USB]{Universidad Simón Bolívar}
\date{\today{}}

\begin{document}
\begin{frame}
  \titlepage
\end{frame}

\section{Summer of '16}

\subsection{Megaparsec}
\begin{frame}{Megaparsec}
\framesubtitle{Now you'll know what happend last summer}
\begin{itemize}
  \item Recuperación de errores mejorada
  \item Fue necesario escribir versión monádica de Megaparsec.Expr, ExprM
\end{itemize}
\end{frame}

\subsection{Modulex}
\begin{frame}{Modulex}
\framesubtitle{When 26 files in a single folder become 41 across four}
\begin{itemize}
  \item Árbol Sintáctico Abstracto
  \item Analizador Sintáctico
  \item Generación de Código
\end{itemize}
\end{frame}

\subsection{Shiny new stuff}
\begin{frame}{Shiny new stuff}
\begin{itemize}
  \item Expresión condicional -> ¡Awesome with functions!
  \item Cuantificador (\#), why not?
  \item Funciones y procedimientos recursivos requieren cota (procedimientos no la implementan aún)
  \item Funciones ahora tienen pre- y poscondición
\end{itemize}
\end{frame}

\subsection{Apuntadores}
\begin{frame}{Apuntadores}
\begin{itemize}
  \item (*) de siempre
  \item New, implementado con calloc
  \item Free
\end{itemize}
\end{frame}

\subsection{Generación de Código para TDEPs}
\begin{frame}{Generación de Código para TDEPs}
\begin{itemize}
    \item Variables de tipo
    \item Se prohibe recursión estructural directa
    \item Se verifica correspondencia entre ``abstracto'' y ``concreto''
    \item Acceso a campos
    \item ¿Implementar operador (->)?
\end{itemize}
\end{frame}

\subsection{Tweaks}
\begin{frame}{Tweaks}
\begin{itemize}
  \item  Producción para Programa mejorada (gramática)
    % \scriptsize
    % \begin{grammar}

    % <Program> ::= `program' <ProgramId>  `begin' <PSDTorRoutines> <MainBlock> <PSDTorRoutines> `end'

    % <ProgramId> ::= <Id> `.' <ProgramId> | <Id>

    % <MainBlock> ::= `main' <Block>

    % \end{grammar}
  \item Flags!
    \begin {itemize}
      \item -S para generación de Assembler
      \item -L para generación de LLVM
      \item -On para optimización
    \end {itemize}
  \item Fallo de postcondición luego de fallo de precondición sólo advierte.
  \item Instrucción `warn`
  \item SRR: String Redundance Reduction
  \item El archivo compilado toma el nombre del program
\end{itemize}
\end{frame}

\subsection{Cuantificadores}
\begin{frame}{Cuantificadores}
\begin{itemize}
  \item El rango puede ser cualquier expresión booleana en forma normal
    conjuntiva que produzca un rango acotado, un rango vacío,
    un rango ``puntual'', o una iteración sobre un conjunto o tipos similares
  \item Spoiler: básicamente son bucles acotados.
\end{itemize}
\end{frame}

\section{Christmas Compiler}

\begin{frame}
\LARGE{Tercera etapa}
\end{frame}

\subsection{Más generación de código para TDEP}
\begin{frame}{Más generación de código para TDEP}
\begin{itemize}
  \item Semántica precisa de las aserciones/invariantes en los TDEP
  \item ¿Variables en procedimientos de TDEPs ``abstractos''?
  \item ¿Funciones en TDEPs?
\end{itemize}
\end{frame}

\subsection{Generación de código para (bi)functores}
\begin{frame}{Generación de código para (bi)functores}
\begin{itemize}
  \item Implementados como librería dinámica en C++
  \item Dudas recolección de basura
\end{itemize}
\end{frame}


\subsection{Idea: Aserciones con mensajes}
\begin{frame}{Idea: Aserciones con mensajes}
\begin{itemize}
  \item \texttt{\{ x > 4, ``acabo de asignar x := 3'' \}}
\end{itemize}
\end{frame}

\end{document}
