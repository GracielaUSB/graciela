%% LyX 2.0.5 created this file.  For more info, see http://www.lyx.org/.
%% Do not edit unless you really know what you are doing.
\documentclass[12pt,oneside,english,spanish]{book}
\usepackage[T1]{fontenc}
\usepackage[utf8]{inputenc}
\usepackage[letterpaper]{geometry}
\geometry{verbose,tmargin=2cm,bmargin=2cm,lmargin=3cm,rmargin=2cm}
\usepackage{fancyhdr}
\pagestyle{fancy}
\setcounter{secnumdepth}{3}
\usepackage{babel}
\addto\shorthandsspanish{\spanishdeactivate{~<>}}

\usepackage{array}
\usepackage{longtable}
\usepackage{graphicx}
\usepackage{setspace}
\onehalfspacing
\usepackage[unicode=true,
 bookmarks=true,bookmarksnumbered=false,bookmarksopen=false,
 breaklinks=false,pdfborder={0 0 1},backref=section,colorlinks=false]
 {hyperref}
\hypersetup{pdftitle={Sistema de Atención de Consultas Médicas},
 pdfauthor={Miguel Miguel Ambrosio Gómez},
 pdfkeywords={USB, Pasantía Larga, Venetecnología, Consultas Médicas,  Agile UP, MVC, Cliente-Servidor, Aplicación Web, CakePHP, PHP, jQuery.}}

\makeatletter

%%%%%%%%%%%%%%%%%%%%%%%%%%%%%% LyX specific LaTeX commands.
\newcommand{\noun}[1]{\textsc{#1}}
%% Because html converters don't know tabularnewline
\providecommand{\tabularnewline}{\\}

%%%%%%%%%%%%%%%%%%%%%%%%%%%%%% User specified LaTeX commands.
 \usepackage{titlesec}
\newcommand*{\justifyheading}{\raggedright}

\titleformat{\chapter}[display]
 {\normalfont\Large\bfseries\centering}
 {\chaptertitlename\ \thechapter}{0pt}{\large}
\titlespacing*{\chapter}{0pt}{-30pt}{20pt}

\titleformat{\section}
 {\normalfont\normalsize\bfseries\justifyheading}
 {\thesection}{1em}{}

\titleformat{\subsection}
 {\normalfont\normalsize\bfseries\justifyheading}
 {\thesubsection}{1em}{}

\titleformat{\subsubsection}
 {\normalfont\normalsize\bfseries\justifyheading}
 {\thesubsubsection}{1em}{}

\usepackage{pdfpages}
\pretolerance=2000
\tolerance=3000

\makeatother

\begin{document}
\pagestyle{empty}

\begin{center}
%%\includegraphics[scale=0.5]{Images/cebolla}
\par\end{center}

\noindent \begin{center}
\textbf{\noun{\large UNIVERSIDAD SIMÓN BOLÍVAR}}\textbf{\noun{}}\\
\textbf{\noun{DECANATO DE ESTUDIOS PROFESIONALES}}\\
\textbf{\noun{COORDINACIÓN DE INGENIERÍA DE LA COMPUTACIÓN}}\noun{}\\
\bigskip{}
\bigskip{}
\bigskip{}
\bigskip{}
\bigskip{}
\bigskip{}
\bigskip{}
\bigskip{}
\bigskip{}
\bigskip{}
\bigskip{}
\bigskip{}

\par\end{center}

\noindent \begin{center}
\textbf{\large IMPLEMENTACIÓN DE UN COMPONENTE PUBLICITARIO (CLIENTE)
PARA APLICACIONES MÓVILES}
\par\end{center}{\large \par}

\bigskip{}
\bigskip{}
\bigskip{}
\bigskip{}
\bigskip{}
\bigskip{}


\begin{center}
Por:\\
Sabrina María Fernández Mirás
\par\end{center}

\bigskip{}
\bigskip{}


\bigskip{}
\bigskip{}
\bigskip{}
\bigskip{}
\bigskip{}
\bigskip{}


\begin{center}
\textbf{\large INFORME DE PASANTÍA}
\par\end{center}{\large \par}

\begin{center}
Presentado ante la Ilustre Universidad Simón Bolívar\\
como requisito parcial para optar por el título de\\
Ingeniero en Computación
\par\end{center}

\bigskip{}
\bigskip{}
\bigskip{}
\bigskip{}
\bigskip{}
\bigskip{}
\bigskip{}
\bigskip{}


\vfill{}


\begin{center}
\textbf{Sartenejas, Septiembre de 2012}
\par\end{center}


\cleardoublepage{}

\begin{center}
%%\includegraphics[scale=0.5]{Images/cebolla}
\par\end{center}

\noindent \begin{center}
\textbf{\noun{\large UNIVERSIDAD SIMÓN BOLÍVAR}}\textbf{\noun{}}\\
\textbf{\noun{DECANATO DE ESTUDIOS PROFESIONALES}}\\
\textbf{\noun{COORDINACIÓN DE INGENIERÍA DE LA COMPUTACIÓN}}\noun{}\\
\bigskip{}
\bigskip{}
\bigskip{}
\bigskip{}
\bigskip{}
\bigskip{}
\bigskip{}
\bigskip{}
\bigskip{}
\bigskip{}

\par\end{center}

\noindent \begin{center}
\textbf{\large IMPLEMENTACIÓN DE UN COMPONENTE PUBLICITARIO (CLIENTE)
PARA APLICACIONES MÓVILES}
\par\end{center}{\large \par}

\bigskip{}
\bigskip{}
\bigskip{}
\bigskip{}
\bigskip{}


\begin{center}
Por:\\
Sabrina María Fernández Mirás
\par\end{center}

\bigskip{}
\bigskip{}
\bigskip{}


\begin{center}
Realizado con la asesoría de:\\
Tutor Académico: Prof. Ángela Di Serio\\
Tutor Industrial: Ing. Carlos Marcano
\par\end{center}

\bigskip{}
\bigskip{}
\bigskip{}
\bigskip{}
\bigskip{}
\bigskip{}
\bigskip{}


\begin{center}
\textbf{\large INFORME DE PASANTÍA}
\par\end{center}{\large \par}

\begin{center}
Presentado ante la Ilustre Universidad Simón Bolívar\\
como requisito parcial para optar por el título de\\
Ingeniero en Computación
\par\end{center}

\bigskip{}
\bigskip{}
\bigskip{}
\bigskip{}
\bigskip{}


\vfill{}


\begin{center}
\textbf{Sartenejas, Septiembre de 2012}
\par\end{center}


\cleardoublepage{}

\begin{center}
%%\includegraphics[scale=0.5]{Images/cebolla}
\par\end{center}

\begin{center}
\textbf{\noun{\large UNIVERSIDAD SIMÓN BOLÍVAR}}\textbf{\noun{}}\\
\textbf{\noun{DECANATO DE ESTUDIOS PROFESIONALES}}\\
\textbf{\noun{COORDINACIÓN DE INGENIERÍA DE LA COMPUTACIÓN}}\\
\bigskip{}
\bigskip{}

\par\end{center}

\begin{center}
\bigskip{}

\par\end{center}

\noindent \begin{center}
\textbf{ACTA FINAL PASANTÍA LARGA}
\par\end{center}

\bigskip{}


\noindent \begin{center}
\textbf{\large IMPLEMENTACIÓN DE UN COMPONENTE PUBLICITARIO (CLIENTE)
PARA APLICACIONES MÓVILES}
\par\end{center}{\large \par}

\bigskip{}


\begin{center}
Presentado por:\\
Sabrina María Fernández Mirás
\par\end{center}

\bigskip{}


\begin{center}
Esta Pasantía Larga ha sido aprobada por el siguiente jurado examinador:
\par\end{center}

\bigskip{}


\begin{center}
\line(1,0){200}\\
Prof. \_\_\_\_\_\_\_\_\_\_\_\_\\
\textbf{Jurado}
\par\end{center}

\bigskip{}


\begin{center}
\line(1,0){200}\\
Prof. Ángela Di Serio\\
\textbf{Tutor Académico} 
\par\end{center}

\bigskip{}


\begin{center}
\line(1,0){200}\\
Ing. Carlos Marcano\\
\textbf{Tutor Industrial} 
\par\end{center}

\bigskip{}


\vfill{}


\begin{center}
\textbf{Sartenejas, 01/09/2012}
\par\end{center}




\cleardoublepage{}

\pagestyle{fancyplain}

\fancypagestyle{plain}{ %
\fancyhf{} % remove everything
\cfoot{\thepage}
\renewcommand{\headrulewidth}{0pt} % remove lines as well
\renewcommand{\footrulewidth}{0pt}}


\pagestyle{plain}

\renewcommand{\thepage}{\Roman{page}}

\setcounter{page}{4} 

\begin{center}
\textbf{\large Resumen}
\par\end{center}{\large \par}

\vspace{5mm}


El presente proyecto de pasantía consiste en el diseño y la implementación
de un componente publicitario para ser incluído en las aplicaciones
desarrolladas para los sistemas operativos móviles iOS y Android.
La empresa Mobile Media Networks C.A. tiene como objetivo ampliar
su alcance publicitario correspondiente a la colocacion de campañas
en aplicaciones móviles. Gracias al éxito de ventas de campañas publicitarias
con la plataforma Blackberry, se desea llegar a los usuarios de otras
tecnologías que están siendo cada vez más populares en Venezuela.
Estas son iOS por la empresa Apple y Android de la empresa Google.
Por lo tanto es necesario la creación de un componente nativo por
cada plataforma que permita incluir de manera sencilla la publicidad
en las aplicaciones.

El componente actual, creado para aplicaciones de Blackberry, se comunica
con el servidor mediante un protocolo que genera una respuesta muy
pesada con información inutil para la colocación de la piblicidad.
Por lo tanto el primer paso es el diseño de un nuevo protocolo que
produzca una respuesta de menor tamaño y con la información mínima
necesaria, lo que tendrá un impacto favorable tanto en espacio como
en tiempo. Una vez tomada la decisión sobre la comunicación con el
servidor, se realizará el diseño de los componentes, analizando el
actual y considerando las posibles mejoras. El componente trabaja
en conjunto con el AdServer MobileX, un servidor que se encarga de
enviar la publicidad de manera inteligente. La realización de los
cambios en el servidor para implementar la nueva comunicación no está
en el alcance de este proyecto.

El componente debe ser capaz de colocar en cualquier lugar de la pantalla
del dispositivo la publicidad, sin importar el tipo de la imagen o
su tamaño original. Además debe reportar al servidor cada vez que
se mostró o que el usuario hizo clic sobre ella. La entrega oportuna
de esta información es crucial para la venta apropiada de campañas
a los clientes y la generación de reportes de las mismas. Finalmente,
debe manejar problemas como fallas en la conexión, el almacenamiento
o la posible sobrecarga del servidor, así como algunas necesidades
nuevas que permiten segmentar la publicidad por país o por tipo de
usuario. 

\begin{onehalfspace}
La metodología utilizada para la implementación de este proyecto es
AUP, una metodología de tipo ágil, con un buen balance entre la planificación
del proyecto y tiempos de desarrollo del software.\newpage{}\end{onehalfspace}



\cleardoublepage{}

\renewcommand{\thepage}{\Roman{page}}

\setcounter{page}{5} 

\vspace*{4cm}
 

\begin{center}
{\large DEDICATORIA}
\par\end{center}{\large \par}

\vspace*{4cm}


\noindent \begin{flushright}
\textit{A mis padres,}\\
\textit{por su apoyo incondicional y su amor. }\\
\textit{Gracias a ustedes he cumplido mis metas.}\\
\textit{A mis hermanas,}\\
\textit{por estar siempre conmigo}\\
\textit{y ser mis amigas más queridas.}\\
\vfill{}

\par\end{flushright}

{\large \newpage{}}


\cleardoublepage{}

\selectlanguage{english}%
\pagestyle{empty}

\selectlanguage{spanish}%
\vspace*{3cm}


\textbf{\LARGE Agradecimientos}{\LARGE \par}

~

\textsl{Gracias a Mobile Media Networks C.A. por brindarme esta oportunidad
única. Ser escogida para formar parte de un equipo de trabajo tan
importante fue un honor para mí. En especial a los Ings. Miguel Sucre
y Ricardo Blanch, su visión de futuro y emprendimiento marcaron la
diferencia. También a mis compañeros de trabajo María Gabriela, Alechandrina,
Ariana, Ricardo, Verónica, Yessenia, Cesar, Ciuffi, Alicia, Andrés,
Edgardo, José Eduardo y al resto de las personas que forman parte
del equipo de Mobile Media, quienes me brindaron un ambiente de trabajo
ameno, divertido y lleno de conocimientos. }

\textsl{A mi tutor industrial, el Ing. Carlos Marcano, por su paciencia
y apoyo incondicional. Sus consejos y ayuda hicieron posible que el
proyecto cumpliera las expectativas. A mi tutora académica, la Profesora
Ángela Di Serio, por su asesoría en el desarrollo del proyecto, siempre
buscando que use todos los conocimientos aprendidos a lo largo de
mi carrera. Nunca dudó que lograría cumplir con todos los objetivos. }

\textsl{A mis compañeros de carrera Miguel, Daniel, Manuel, Christian,
Daniela, Fabio, Ricardo, María Gracia, y Fernando, con los cuales
compartí buenos y malos momentos. A Mariana Bezada, que tanto tiene
que enseñarme todos los días, siempre la primera en ayudarme. A Mayerlin
González, por hacer de la Universidad una experiencia divertida. A
Jonathan, porque aunque nos separe todo un océano siempre estás conmigo
para asesorarme, nunca has dejado de estar pendiente de mí. Y no podía
faltar Mariana Rodríguez, por ser mi tercera hermana. Dieciocho años
de amistad han sido pocos con respecto a los que nos faltan. }

\textsl{A mis padres, María Jesús y Arturo, por no dejarme olvidar
mi meta. Han sido mi apoyo y mi mayor fortaleza. Gracias por su amor
incondicional a pesar de mi difícil carácter. A mi hermana Cristina,
quien a pesar de ser menor que yo todos los días tiene una lección
de madurez que darme. A mi hermana Mónica, por siempre tener una sonrisa
para mi. A mi abuela María Dolores, y a todos los que no he mencionado
aquí, que aunque no estén lejos, siempre los tengo cerca de mí corazón. }

\textsl{A la Universidad Simón Bolívar, por convertirse en mi segunda
casa. }

\textsl{A Dios, por darme fuerzas y esperanza.}

\textsl{A todos ustedes, mil gracias.}


\cleardoublepage{}

\pagestyle{fancy}

\addcontentsline{toc}{chapter}{Índice general}\tableofcontents{}

\cleardoublepage{}

\selectlanguage{english}%
\addcontentsline{toc}{chapter}{Lista de tablas}

\selectlanguage{spanish}%
\listoftables


\cleardoublepage{}

\addcontentsline{toc}{chapter}{Lista de figuras}

\listoffigures


\cleardoublepage{}

\pagestyle{fancy}


\chapter*{{\large Lista de Símbolos y Abreviaturas}}

\renewcommand{\thepage}{\Roman{page}}

\selectlanguage{english}%
\addcontentsline{toc}{chapter}{Lista de Símbolos y Abreviaturas}

\selectlanguage{spanish}%
\vspace{5mm}


\begin{onehalfspace}
\begin{center}
\begin{tabular}{llll>{\raggedright}p{11cm}}
API &  &  &  & Application Programming Interface\tabularnewline
AUP &  &  &  & Agile Unified Process\tabularnewline
HTTP &  &  &  & Hypertext Transfer Protocol\tabularnewline
IDE &  &  &  & Integrated Development Enviroment\tabularnewline
IP &  &  &  & Internet Protocol\tabularnewline
JSON &  &  &  & JavaScript Object Notation\tabularnewline
MCC &  &  &  & Mobile Country Code\tabularnewline
MNC &  &  &  & Mobile Network Code\tabularnewline
RPC &  &  &  & Remote Procedure Call\tabularnewline
RUP &  &  &  & Rational Unified Process\tabularnewline
XML &  &  &  & eXtensible Markup Language\tabularnewline
\end{tabular}
\par\end{center}
\end{onehalfspace}

\pagebreak{}


\cleardoublepage{}

\pagestyle{myheadings}

\markright{}

\selectlanguage{english}%
\renewcommand{\thepage}{\arabic{page}}

\setcounter{page}{1}

\setlength{\parskip}{7.5pt}

\selectlanguage{spanish}%

\chapter*{{\large Introducción}}

\selectlanguage{english}%
\addcontentsline{toc}{chapter}{Introducción}\foreignlanguage{spanish}{\thispagestyle{empty} }

\selectlanguage{spanish}%
El mundo de la tecnología cambia día a día, donde los usuarios experimentan
continuamente con una gran cantidad de formas de tecnologías creativas
e innovadoras. Cada semana salen al mercado aplicaciones móviles o
portales web que ofrecen servicios que hacen más fácil y entretenida
la vida de las personas, tales como las tiendas y bancos virtuales,
el correo electrónico y los mensajes instantáneos.

Es así que el consumo crece en igual medida, por lo que comunicar
en masa a través de este medio es de gran importancia para las empresas
que deseen dar a conocer sus productos o servicios. 

La colocación de publicidad en los dispositivos móviles es un suceso
reciente que ha dado buenos resultados. Mobile Media Networks se especializa
en esta área, siendo uno de sus servicios ofrecidos la distribución
de publicidad en aplicaciones de dispositivos móviles. Esta distribución
consiste en la colocación de la publicidad y en la presentación de
los reportes a los anunciantes. El reporte consta de datos importantes,
como impresiones, interacciones y el porcentaje de efectividad. Sin
embargo los métodos de mercado se hacen cada día más exigentes y ambiciosos
con la idea de promover acciones en los usuarios de la publicidad.
Los anunciantes desean dirigir su publicidad a quién muestre interés
por ella, a quien le sea relevante.

La empresa utiliza para la distribución de publicidad en los dispositivos
móviles, el componente AdField MobileX de Blackberry, en conjunto
con el sistema AdServer MobileX, basado en la plataforma OpenX. En
base a las necesidades se ha planteado la creación de los componentes
publicitarios para las plataformas iOS y Android, incluyendo nuevas
funcionalidades que el AdField no contempla actualmente.

El presente proyecto describe el proceso de desarrollo del componente
publicitario AdView MobileX, que permita la colocación de las creatividades
en las aplicaciones móviles, así como el reporte de las impresiones
y clics realizados, el envío de datos importantes de segmentación
al servidor, y el almacenamiento y rotación de las creatividades.
Fue desarrollado bajo la metodología AUP en un período de 20 semanas.

El informe desarrollado para este proyecto de pasantía larga presenta
una estructura en capítulos. El primer capítulo describe el entorno
laboral, para la ubicación del escenario del problema desarrollado.
El segundo capítulo comprende el marco teórico donde se exponen los
conceptos necesarios para la comprensión del problema y de la solución
realizada. El tercer capítulo muestra la metodología utilizada AUP.
El cuarto capítulo presenta el marco tecnológico describiendo las
herramientas utilizadas para el desarrollo del proyecto. En el capítulo
quinto se describe el desarrollo del proyecto y los resultados basado
en la metodología usada. Finalmente se plantean las conclusiones y
recomendaciones.

A continuación, se exponen los objetivos generales y específicos que
se busca alcanzar en este desarrollo, y el alcance que abarca este
proyecto. Esta información tiene la intención de que el lector comprenda
el contexto general del proyecto y conozca el propósito del mismo.

\vspace{7.5pt}


\noindent \textbf{Objetivo general}\vspace{7.5pt}


Diseñar un protocolo eficiente y un componente nativo para cada plataforma
(IOS y Android) que permita entregar y reportar los impactos publicitarios
asignados.\vspace{7.5pt}


\noindent \textbf{Objetivos específicos}\vspace{7.5pt}

\begin{itemize}
\item Diseñar un protocolo que permita enviar toda la información necesaria
para mostrar la publicidad y reportar de vuelta la correcta entrega
de los datos y las acciones tomadas por el usuario.
\item Diseñar un componente que se acople con el protocolo planteado y permita
la recepción y visualización de la información enviada por el AdServerMobileX.
\item Desarrollo de un componente nativo IOS a partir del diseño propuesto.
\item Desarrollo de un componente nativo Android a partir del diseño propuesto.
\end{itemize}
\vspace{7.5pt}


\noindent \textbf{Alcance}

\vspace{7.5pt}


El alcance de este proyecto es el desarrollo de librerías para la
introducción de publicidad interactiva en las aplicaciones nativas
con el objetivo de ofrecer una presentación personalizada. La entrega
del prototipo será 100\% funcional.

\selectlanguage{english}%
\pagebreak{}\selectlanguage{spanish}%




\chapter{Entorno Empresarial}

\thispagestyle{empty} 

Este capítulo da a conocer el entorno empresarial en el cual se desarrolla
el proyecto de pasantías. Busca mostrar la labor de Mobile Media Networks,
dando una idea de sus objetivos, misión y aspectos históricos relevantes. 


\section{Mobile Media Networks}

La empresa Mobile Media Networks se dedica al desarrollo de portales
y aplicaciones en dispositivos móviles, así como la distribución de
publicidad en los mismos. A través del manejo de distintas plataformas
móviles, diseñan y programan aplicaciones que logran integrar el contenido
y la publicidad de distintas empresas a tabletas, celulares y otros
dispositivos portatiles. A continuación se presentará una breve descripción
de la empresa.


\subsection{Reseña Histórica}

Mobile Media Networks nace de la empresa \textquotedblleft{}Adsmedia
Mobile Advertising\textquotedblright{} de España, la cual se dedica
al mercadeo móvil y a la publicidad. Es en Barcelona, en el año 2006,
cuando es fundada Adsmedia Mobile Advertising, como una empresa global,
joven y dinámica con la capacidad de adaptar de forma inmediata la
oferta de productos y servicios publicitarios a las necesidades nuevas
y cambiantes del mercado. 

En el año 2007 gana el premio \textquotedblleft{}Empresa Innovadora\textquotedblright{}
de la Generalitat de Catalunya, el premio \textquotedblleft{}las 100
mejores ideas empresariales\textquotedblright{} otorgado por la revista
Actualidad Económica y queda como finalista en BCN-Emprendedores edición
2007. Al poco tiempo es seleccionada por Nokia como \textquotedblleft{}partner\textquotedblright{}
estratégico para el desarrollo de Publicidad Móvil en España, Portugal
y Latinoamérica. En mayo de este año se abren nuevas oficinas en Londres
y en Caracas.\cite{Cap1.Hitos}

En el 2008 surgen nuevos contratos de gestión y desarrollo con empresas
como Movistar, Vodafone, Orange, Meridiano y Digitel. También tienen
lugar expansiones y nuevas versiones del sistema encargado de la distribución
de publicidad, \textquotedblleft{}Mobile Advertising Adserver\textquotedblright{},
ofreciendo nuevas tecnologías como el soporte de videos. También se
realiza la apertura de oficinas en Brasil. 

En el año siguiente es desarrollada y puesta en uso la nueva versión
del sistema \textquotedblleft{}Mobile Advertising Adserver\textquotedblright{}
y nuevas oficinas son inauguradas en Bahrein y Lisboa. Por otro lado
la empresa tiene presencia en el Mobile World Congress 2009. \cite{Cap1.Hitos}

A mediados del año 2009 ocurre el traspaso de las operaciones de \textquotedblleft{}Adsmedia
Mobile Advertising\textquotedblright{} en España a \textquotedblleft{}Adsmedia
Mobile Advertising\textquotedblright{} en Venezuela, otorgando a estas
oficinas la gestión del conjunto de sucursales distribuidas a nivel
mundial y el control de su desarrollo. Poco después esta empresa adquiere
independencia con respecto a la sucursal en España y en el año 2010
cambian su nombre a \textquotedblleft{}Mobile Media Networks\textquotedblright{}. 

Actualmente Mobile Media Networks posee oficinas en Caracas, Bogotá
y Miami, prestando los servicios de distribución de publicidad en
portales web y dispositivos portátiles, así como el desarrollo de
aplicaciones móviles para las plataformas iOS, Android y Blackberry.


\subsection{Misión}

Mobile Media Networks es la empresa especializada en brindar soluciones
de mercadeo en celulares y publicidad en Latinoamérica y la ciudad
de Miami.


\subsection{Visión}

Mobile Media Networks presenta su visión como: 

\textquotedblleft{}Creer en la liberalización de las comunicaciones
y en el desarrollo de la publicidad en el móvil. El móvil se ha convertido
en un nuevo soporte publicitario, por ello en Mobile Media Networks
se persigue generar valor para toda la cadena implicada, desde los
usuarios hasta las agencias de medios, pasando por los operadores
y los anunciantes. 

Incorporar publicidad a toda la cadena de comunicación, con lo que
se incrementa el acceso de los anunciantes a las comunicaciones, generando
servicios gratuitos y, lo más importante, reduciendo los costes para
el usuario.\textquotedblright{}\cite{Cap1.AdsMedia}


\subsection{Estructura organizacional}

En la Figura 1.1, se muestra la estructura organizacional de Mobile
Media Networks.

.

\vspace*{1cm}


\begin{center}
\begin{figure}[h]
\centering{}%%\includegraphics[scale=0.6]{\string"Images/Organigrama de la Empresa\string".png}\caption{Estructura organizacional MobileMedia Networks C.A.}
\end{figure}

\par\end{center}

La realización del proyecto de pasantía fue bajo el cargo de \textquoteleft{}Pasante\textquoteright{}
en la dirección de Tecnología de Mobile Media Networks.


\subsubsection{Resumen del Cargo de Pasante de Tecnología}

El pasante de Tecnología es el estudiante que busca conocimientos
que lo ayuden a desarrollarse como profesional. Las pasantías académicas
son las que representan un requisito para graduarse y se caracteriza
por tener tiempos académicos y proyectos específicos.\cite{Cap1.Mobile}
\newpage{}



\chapter{Marco Teórico}

\thispagestyle{empty} 

A continuación, se presentan los conceptos teóricos relevantes sobre
los cuales se fundamentó el proyecto de pasantía, que incluyen el
marco de trabajo y las herramientas utilizadas.


\section{Conceptos Básicos}


\subsection{Mobile X}

Es una herramienta para la gestión de campañas publicitarias en dispositivos
móviles basado en OpenX. MobileX cumple con el modelo cliente-servidor.
El cliente es el componente encargado de mostrar la publicidad en
un portal web o en una aplicación, mientras que el servidor es el
responsable de proveer al cliente la publicidad adecuada, la cual
se identifica mediante un número llamado identificador de zona (zoneId). 

La comunicación entre ambos se realiza mediante el protocolo XML-RPC
para la versión 1.0 del cliente, y el protocolo JSON-RPC para la versión
2.0.


\subsubsection{AdServer MobileX}

Es una plataforma publicitaria para la entrega de publicidad en dispositivos
móviles, basado en el AdServer de OpenX. Posee características para
el manejo y la distribución de campañas y un sistema de rastreo para
presentar las estadísticas.

Está licenciado bajo GNU General Public License y escrito en su mayor
parte en PHP, utilizando MySQL como manejador de base de datos. La
instalación esta hecha sobre los servidores de Amazon, bajo el servicio
web de computación de nube (Cloud Computing) Amazon EC2 (Elastic Cloud
Computing).

El AdServer maneja varios módulos para la entrega y el reporte de
las campañas. Una campaña es un conjunto específico de banners, los
cuales tienen una fecha de inicio y de finalización, así como una
frecuencia con la que serán mostrados. Se manejan tres tipos de campaña:
\begin{enumerate}
\item Exclusiva: es aquella campaña que se ejecuta antes que cualquier otra. 
\item De contrato: es aquella campaña que será entregada un número específico
de veces por días, por un número específico de días. 
\item Remanente: es aquella campana que se mostrará luego de las anteriores,
limitada por un número de impresiones o una fecha de finalización. 
\end{enumerate}
Un banner es cualquier tipo de contenido creativo que será mostrado
como una publicidad. Puede ser de formato JPG, PNG, GIF y SWF, así
como ser escrito en Javascript, HTML. Un banner para una campaña puede
tener varias versiones por los distintos tamaños predeterminados de
los dispositivos actuales para los cuales será distribuida la publicidad.


\subsubsection{AdField MobileX}

Es un componente publicitario que actúa como cliente para la plataforma
Blackberry cuya finalidad es mostrar las campañas publicitarias en
los dispositivos, así como reportar la impresión y el clic de las
mismas. Esta escrito completamente en java, utilizando algunas librerías
de RIM. Su nombre se debe a que en Blackberry un área rectangular
se denomina un campo (field), por lo que una creatividad se adapta
a esta definición.

Actualmente se encuentra operativa la versión 1.0 del componente y
en desarrollo la versión 2.0 en base a las nuevas funcionalidades
y cambios propuestos en este proyecto de pasantía. 

Para poder desplegar el contenido publicitario, el cliente debe realizar
una solicitud al servidor enviando un número de zona. La zona identifica
un espacio donde las formas creativas son desplegadas. Cuando el servidor
recibe el número de zona, ejecuta un código de invocación, el cual
retorna la información del banner necesaria para ser mostrado.

La información mínima que se debe tener para poder desplegar el banner
es:
\begin{enumerate}
\item Dirección de contenido: dirección url donde se encuentra la creatividad. 
\item Dirección de impresión: dirección url que será llamada cuando se muestre
la creatividad en la aplicación.
\item Dirección de clic: dirección url que será llamada cuando se haga clic
sobre la creatividad en la aplicación. 
\item Ancho: ancho que mide la creatividad. (Debe ser menor que el ancho
de la pantalla del dispositivo). 
\item Alto : alto que mide la creatividad. (Debe ser menor que el alto de
la pantalla del dispositivo). 
\item Identificador de banner: identificador del tipo de creatividad. 
\item Tipo de contenido: define si la creatividad es de formato jpg, png,
gif, entre otros posibles.
\item Tipo de campaña: define si la campaña es exclusiva, de contrato o
remanente.
\end{enumerate}

\subsubsection{AdView MobileX}

Es un componente publicitario que actúa como cliente para las plataformas
iOS y Android. Este componente cumple con las mismas funcionalidades
que el componente AdField MobileX, solo que adopta este nombre porque
un área rectangular en estas plataformas se denomina una vista (view),
término bajo el cual se adapta la definición de una creatividad. 

La versión de Android está escrito completamente en Java, utilizando
algunas librerías de propias de Android, mientras que la versión de
iOS está escrito en Objective-C. Para ambas versiones fue necesaria
la utilización de algunas librerías de terceros para implementar algunas
funcionalidades.

Actualmente se encuentra operativa la versión 2.0 del componente,
incluyendo los cambios pertinentes del servidor.

Para esta versión el cliente debe realizar una solicitud al servidor
enviando un número de zona, el tamaño de la pantalla (ancho y alto),
un identificador del registro (cookie), el Código de Area de País
(Mobile country code) y el Código de Area del Operador de Telefonia
(Mobile network code).


\section{Herramientas}


\subsection{Modelo Cliente - Servidor}

La arquitectura cliente-servidor permite al usuario en un dispositivo
movil, llamado el cliente, requerir algún servicio de una máquina
a la que está unido, llamado el servidor, mediante una red, ya sea
de datos o inalamrbica (Wi-Fi). Estos servicios pueden ser peticiones
de datos de una base de datos, de información contenida en archivos
o los archivos en sí mismos, o peticiones de imprimir datos en una
impresora asociada~\cite{Cap2.ClienteServidor}.

El modelo cliente-servidor se fundamenta en la idea de que un proveedor
de datos por si mismo puede abastecer las solicitudes de numerosos
clientes que requieren de sus servicios.

El componente es un ejemplo de un sistema basado en este modelo. Cada
cliente estará representado por el componente publicitario, que realizará
solicitudes al servidor en el momento oportuno solicitando una creatividad;
y el servidor, que contiene la aplicación de gestión de solicitudes
de publicidad, atenderá cada una de las solicitudes del cliente.


\subsection{Agente de Usuario}

El agente de usuario es una cadena de caracteres que identifica a
un dispositivo o un portal web. Los datos que incluye son:
\begin{enumerate}
\item El nombre de la aplicación y la versión.
\item El tipo de explorador y su versión.
\item El sistema operativo.
\item Las extensiones instaladas en el sistema o en el explorador.
\end{enumerate}
\begin{figure}[h]
%%\includegraphics{Images/useragent}

\caption{Formato del agente de usuario \cite{Cap2.Useragent}}
\end{figure}



\subsection{Librerías de Terceros}

Las librerías de terceros (Third-party software component) son componentes
desarrollados para implementar alguna funcionalidad que no esté presente
en la plataforma que se trabaja o para mejorar las que existan. Pueden
ser vendidas o distribuidas libremente, por lo que cada librería posee
una licencia asociada límita su uso. La ventaja de estas librerías
es que el desarrollador no tiene que estar escribiendo un nuevo componente,
pero tiene que estar conciente de que existe la posibilidad de que
el componente presente errores que el creador original no manejó.



\chapter{Marco Metodológico}

\thispagestyle{empty} 

Este capítulo presenta los aspectos metodológicos para el desarrollo
del proyecto de pasantía, se realiza una breve descripción la metodología
utilizada así como su implementación en este proyecto.


\section{AUP}

El Proceso Unificado Agil (\textit{Agile Unified Process}, AUP o AgileUP)
es una metodología para el desarrollo de software, basado en el Rational
Unified Process (RUP) de IBM. Describe de una forma simple la forma
de desarrollar aplicaciones de software de negocio utilizando técnicas
ágiles y conceptos que aún se mantienen válidos en RUP. \cite{Cap3.AUP2}

AUP emplea técnicas ágiles, incluyendo Desarrollo Guiado por Pruebas
(\textit{Test Driven Development} o TDD), modelado ágil, gestión ágil
de cambios y refactorización de bases de datos. Algunas de estas técnicas
formaron parte integral del desarrollo del proyecto. 

AUP establece metas y objetivos donde las colaboraciones son usadas
para lograr resultados. Una colaboración puede referirse a un trabajo
realizado por un sólo individuo que interactúa con otros, o puede
referirse a un grupo de trabajo que involucra a múltiples individuos
que interactúan unos con otros y con agentes externos. \cite{Cap3.AUP1} 

Para el desarrollo de este proyecto la metodología AUP fue implementada
siguiendo la documentación de la versión 1.1 del Proceso Unificado
Ágil. Se emplea AUP debido a que presenta un equilibrio entre las
etapas de diseño, planificación y documentación del sistema, con la
implementación del mismo, por lo tanto, se sigue manteniendo un desarrollo
estructurado y coherente, pero brindando una mayor flexibilidad para
el programador, tanto para en tiempo efectivo de implementación, como
en el manejo de cambios en el sistema a través del proceso de desarrollo.


\section{Fases de AUP}

La metodología AUP, (al igual que RUP), busca dividir el desarrollo
de software en cuatro (4) fases, cada una de las cuales comprende
técnicas y disciplinas de desarrollo, que permiten iterar sobre ellas
hasta lograr el cumplimiento de los objetivos planteados. Al ser una
metodología flexible y ágil, AUP permite que las técnicas que componen
cada fase, se puedan adecuar a las necesidades y realidades del proyecto. 

A continuación, se muestra un gráfico con las fases de la metodología
AUP (Figura 3.1).

\begin{center}
\begin{figure}[h]
\centering{}%%\includegraphics[scale=0.65]{\string"Images/Fases AUP\string".png}\caption{Fases y disciplinas de la metodología AUP \cite{Cap3.AUPFases}.}
\end{figure}

\par\end{center}

El eje vertical representa la dimensión dinámica, relativa al tiempo
y el eje horizontal la dimensión estática.


\section{Implementacion en el Proyecto de Pasantia}

AUP es una metodología de cuatro fases, cada una compuesta por un
conjunto de técnicas y disciplinas de desarrollo. Por ser una metodología
ágil, estas técnicas y disciplinas pueden ser adaptadas a las necesidades,
recursos e intereses del proyecto y de la empresa.

Durante el desarrollo de este proyecto se llevaron a cabo las cuatro
fases de la metodología: inicio, elaboración, construcción y transición.
Las fases de inicio y elaboración se llevaron a cabo en una iteración,
la fase de construcción en tres iteraciones por cada plataforma y
la fase de transición en una iteración. 


\subsection{Inicio (\textit{Inception})}

Esta es la primera fase del ciclo de vida donde se identifica el alcance
inicial del proyecto. El objetivo principal es archivar el consenso
de los interesados del proyecto en relación a los objetivos del mismo
para obtener financiamiento. \cite{Cap3.AUPFases}

En el contexto de este proyecto los objetivos dispuestos para esta
fase están centrados en la definición de los requerimientos del sistema.
Las actividades realizadas comprenden la introducción a la empresa,
investigación acerca de la situación actual, obtención y modelación
inicial de los requerimientos, planteamiento y estudio de las herramientas
a utilizar, elaboración de un plan de riesgos, elaboración inicial
de un plan de pruebas, elaboración inicial del glosario y la planificación
de actividades para las siguientes iteraciones. 


\subsection{Elaboración (\textit{Elaboration})}

La segunda fase consiste en probar la arquitectura potencial del sistema.
Se trata de determinar que el equipo pueda desarrollar un sistema
que satisfaga los requisitos, por lo que se realiza una construcción
de extremo a extremo del esqueleto del trabajo, conocido como \textquotedblleft{}prototipo
de la arquitectura\textquotedblright{}. 

Los riesgos se detallan lo suficiente como para entender los riesgos
de la arquitectura y para asegurar que exista una comprensión de los
alcances de cada requerimiento para que la planificación posterior
se puede llevar a cabo. {[}19{]}

En este proyecto las actividades comprendidas durante esta fase involucran
la identificación de los riesgos técnicos, el modelado de la arquitectura
y diseños de prototipos, así como la refinación de documentos desarrollados
en la fase anterior. 


\subsection{Construcción (\textit{Construction})}

Esta fase involucra la construcción de un software funcional sobre
una base regular e incremental, las cuales cumplan con las prioridades
más importantes para los involucrados o usuarios del proyecto. El
objetivo de esta fase consiste en desarrollar el sistema hasta el
punto que se encuentra listo para la pre-producción de pruebas. {[}19{]} 

Para este proyecto esta fase presenta tres iteraciones: 
\begin{itemize}
\item Primera iteración:

\begin{itemize}
\item Diseño de un nuevo protocolo de comunicacion con el servidor.
\item Diseño e implementación del mecanismo de obtención de datos.
\end{itemize}
\item Segunda iteración:

\begin{itemize}
\item Diseño e implementación de almacenamiento en cache.
\item Diseño e implementación del funcionamiento al no tener red de datos.
\end{itemize}
\item Tercera iteración:

\begin{itemize}
\item Diseño e implementación de la lógica de retraso de peticiones.
\end{itemize}
\end{itemize}

\subsection{Transición (\textit{Transition})}

Esta última fase consiste en la validación y despliegue del sistema
en su ambiente de producción. Durante esta fase deben hacerse pruebas
extensivas, para asegurarse de que el sistema puede ser desplegado
de manera segura y eficiente. {[}19{]}

En el desarrollo de este proyecto en la fase de transición se realizaron
las pruebas necesarias para comprobar el funcionamiento correcto de
los componentes para las dos plataformas y la refinación de documentos.
El alcance de este proyecto de pasantía sólo abarca hasta la generación
de documentos, con la elaboración del informe de pasantía. En este
sentido el proceso de puesta en producción queda fuera de las 20 semanas
estipuladas. 

\newpage{}



\chapter{Marco Tecnológico}

\thispagestyle{empty} 

A continuación, se presentan los aspectos tecnológicos para el desarrollo
del proyecto de pasantía, se describen brevemente y se indican su
uso como parte de la solución propuesta. 


\section{Android}

Es un sistema operativo para dispositivos móviles tanto teléfonos
celulares como tabletas. Fue desarrollado por Android Inc, firma que
fue comprada por Google en el año 2005. Es el más utilizado en el
mundo, esto a razón del bajo costo que ofrece a las empresas fabricantes
de los dispositivos. Además Android también es personalizable y ligero,
cualidades muy importantes y atractivas. La mayoría del código de
Android está liberado bajo la licencia Apache.

El sistema operativo Android consta de un núcleo basado en los núcleos
de Linux 2.6 y 3.x, con librerías, APIs y middleware escritos en C.
El software de aplicación se ejecuta sobre un framework que incluye
librerías compatibles con Java basadas en Apache Harmony. Usa Dalvik
como máquina virtual, compilando las aplicaciones en tiempo de ejecución.Android
usa OpenCore como Framework, SQLite como manejador de base de datos,
WebKit como motor de renderizado, OpenGL como interfaz de programación
de API gráfica, un administrador de interfaz gráfica, un motor gráfico
SGL, una biblioteca estándar de C llamada Bionic y una capa de conexión
segura (SSL).\cite{Cap4.Android}

La última versión es la 4.2.1, llamada Jelly Bean (correspondiente
al API 17). Salió al mercado en noviembre del año 2012.

Las aplicaciones, tanto pagas como gratuitas, pueden ser descargadas
de la tienda virtual Play Store, mediante una aplicación con el mismo
nombre.


\subsection{Java}

Es un lenguaje de programación de alto nivel orientado a objetos desarrollado
por Sun Microsystems. Las aplicaciones escritas en Java son compiladas
a bytecode y pueden ser ejecutadas en cualquier máquina virtual de
Java. Gracias a su portabilidad, Java es usado para el desarrollo
de aplicaciones móviles para las plataformas Blackberry y Android.

Android ha creado un conjunto de librerías en este lenguaje que, en
conjunto con las librerías propias de Java, permiten el desarrollo
de aplicaciones para dispositivos portátiles que tengan instalado
el sistema operativo Android. En el portal web http://developer.android.com/reference/packages.html
se puede encontrar la referencia de las librerías que soporta una
aplicación Android, con la facilidad de filtrarlas por nivel de API.

Además de las librerías propias de Java, y las agregadas por Android,
se utilizaron librerías de terceros para poder incluir algunas funcionalidades
que no poseen las versiones de los sistemas operativos que el componente
debe soportar.


\section{iOS}

Es un sistema operativo para dispositivos móviles desarrollado y distribuido
por la compañía Apple, para ser usado únicamente en los dispositivos
fabricados por la misma firma. Entre estos dispositivos se encuentran
el iPhone, iPod Touch, iPad y el Apple TV. 

iOS es un derivado de OS X, el cual es un sistema operativo tipo Unix.
La estructura de iOS está basado en una abstracción de cuatro capas:
la Capa de Núcleo del Sistema Operativo, la Capa de Servicios, la
Capa Multimedia y la Capa de Cocoa Touch (framework de interfaz de
usuario). Está escrito en C, C++ y Objective-C. 

La última versión es la 6.1, disponible desde diciembre del 2012.

Las aplicaciones, tanto pagas como gratuitas, pueden ser descargadas
de la tienda virtual App Store, mediante una aplicación con el mismo
nombre.


\subsection{Objective-C}

Es un lenguaje de programación orientado a objetos de alto nivel.
Su sintaxis es similar a la de Smalltalk. Es usado para el desarrollo
de los sistemas operativos de Apple OS X y iOS, así como para las
Interfaces de Programación de Aplicaciones (APIs) Cocoa y Cocoa Touch.

Solo se permite el desarrollo de aplicaciones en Objective-C mediante
el uso de un IDE llamado Xcode, el cual se puede descargar gratuitamente
si se posee el hardware y software de Apple. La última versión de
Xcode es la 4.5.2. Xcode permite instalar los simuladores de los dispositivos
de las últimas versiones para la realización de las pruebas. La documentación
sobre las librerías para el desarrollo de aplicaciones se encuentra
en el portal web ttps://developer.apple.com/library/ios/navigation/index.html. 


\section{RPC}

La Llamada a Procedimiento Remoto (Remote Procedure Call) es un mecanismo
para las aplicaciones Cliente-Servidor, desarrollado por Sun Microsystems.
En la comunicación participa un procedimiento invocador (cliente)
y un procedimiento invocado (servidor). El objetivo de RPC es que
el cliente realice una llamada a un procedimiento del servidor como
si el mismo sea local. 

En lineas generales, para realizar la llamada a un procedimiento remoto,
el programa cliente debe enlazarse con un procedimiento de librería
llamado \textit{stub} del cliente, que representa el procedimiento
del servidor en el espacio del cliente. De la misma forma, el servidor
se enlaza con un procedimiento llamado \textit{stub} del servidor.
Ambos procedimiento ocultan el hecho de que la llamada no es local. 

La comunicación sconsta de los siguientes pasos:
\begin{enumerate}
\item El cliente llama a \textit{stub} del cliente. Esta es una llamada
local, por lo que los parámetros se colocan en la pila.
\item El \textit{stub }del cliente empaqueta los parámetros en un mensaje
y realiza una llamada de sistema para enviar el mensaje.
\item El kernel envía el mensaje desde la máquina cliente a la máquina servidor.
\item El stub del cliente llama al procedimiento servidor con los parámetros.
\end{enumerate}
La respuesta del servidor al cliente se realiza de la misma forma.
\cite{Cap4.RPC1}


\subsection{XML-RPC }

Es una llamada a procedimiento remoto que usa HTTP como protocolo
de transporte y XML para la codificación de los mensajes. Permite
la transmisión de estructuras de datos complejas, pero la interpretación
de las mismas es sencilla. \cite{Cap4.XMLRPC}

XML-RPC es la forma de comunicación que tiene actualmente el AdServer
MobileX con el componente AdField MobileX. Tanto la petición que hace
el cliente como la respuesta que recibe son empaquetadas en un XML,
y ambas contienen muchos etiquetas que son inútiles tanto para el
cliente como para el servidor.


\subsubsection{XML (eXtensible Markup Language) }

Es un lenguaje desarrollado por el World Wide Web Consortium (W3C)
que es usado para el intercambio de información entre distintas plataformas.
Su flexibilidad permite que sea usado para representar cualquier estructura
de datos, desde arreglos y árboles hasta, por ejemplo, un servicio
web o una base de datos. 


\subsection{JSON-RPC}

Es un protocolo de llamada a procedimiento remoto similar a XML-RPC,
solo que el formato de codificación es JSON en lugar de XML. Utilizar
JSON permite enviar estructuras de datos tan complejas como las que
se pueden enviar con XML-RPC, pero el mensaje es más ligero. \cite{Cap4.JSONRPC}

Una variación del JSON-RPC es la forma de comunicación propuesta para
la comunicación entre el AdServer MobileX y el nuevo componente AdView
MobileX. Los argumentos de la solicitud de publicidad no se empaquetarán
en un JSON, si no que se enviarán como argumentos al hacer la petición
HTTP. La respuesta será un JSON que contenga solo los campos necesarios
para mostrar y reportar la publicidad.


\subsubsection{JSON (JavaScript Object Notation) }

Es un lenguaje que ofrece un formato ligero para el intercambio de
información, siendo una alternativa a XML. Su ventaja frente a XML
es que su formato permite la creación de un diccionario, por lo que
no son necesarias el exceso de banderas que necesita XML para la represantación
de las estructuras. En la figura 4.1 se ven los dos formatos representando
los mismo datos.

\begin{center}
\begin{figure}[h]
\centering{}%%\includegraphics[scale=0.6]{Images/xmlvsjson}\caption{Representación de datos con elos formatos XML y JSON.}
\end{figure}

\par\end{center}

\newpage{}



\chapter{Desarrollo}

\thispagestyle{empty} 

A continuación, se presenta el proceso de desarrollo de la pasantía.
Se describen, de manera específica, cada una de las actividades realizadas
en cada fase de AUP, las dificultades encontradas y las decisiones
tomadas a lo largo del desarrollo. 


\section{Face de Inicio}

Los objetivos principales de la primera fase son la identificación
del alcance inicial del proyecto y la modelación de los requerimientos
especificados para el componente. Esta fase fue planificada para ser
desarrollada en una iteración. 


\subsection{Actividades propuestas dentro de la fase de inicio}

Las actividades consideradas a llevar a cabo en la fase de inicio
son las siguientes:
\begin{enumerate}
\item Introducción a la empresa.
\item Investigación acerca la situación actual, lo que incluye: 

\begin{enumerate}
\item Investigación acerca la publicidad en dispositivos móviles. 
\item Investigación acerca las formas de publicidad entregada por la empresa. 
\item Investigación acerca de componentes de publicidad para móviles en
el mercado.
\item Investigación acerca del AdServer MobileX. 
\item Investigación acerca del AdField MobileX.
\end{enumerate}
\item Modelación inicial de los requerimientos.
\item Planteamiento y estudio de las herramientas a utilizar, así como de
los usuarios involucrados. 
\item Elaboración inicial de un Plan de Riesgos. 
\item Planificación de las actividades para las siguientes iteraciones.
\end{enumerate}

\subsection{Modelado de Requerimientos de Alto Nivel}

El objetivo del modelado de requerimientos es identificar las metas
del negocio, desarrollar una visión común y determinar los requerimientos
iniciales del componente en un alto nivel. 

Para la determinación del alcance inicial del proyecto fue necesaria
la construcción de un listado de requerimientos, con la finalidad
de obtener una visión acerca de lo qué se pretendía lograr y la identificación
de una solución viable. Debido a limitaciones de tiempo y por decisión
de los dirigentes de la empresa donde fue desarrollado el proyecto
de pasantía, el enfoque del desarrollo de este proyecto estuvo dirigido
a requerimientos funcionales.

Luego de establecer los requerimientos, estos se dividieron en tres
categorías, dependiendo de la prioridad y las funcionalidades que
se desean implementar. Las categorías son: Requerimientos Básicos,
Requerimientos Avanzados y Requerimientos Adicionales.

Los Requerimientos Básicos son aquellos que permiten hacer la solicitud
de la publicidad y colocarla adecuadamente en los dispositivos portátiles.
Además permiten hacer el reporte de las impresiones en pantalla y
de los clics hechos soble la publicidad. Los siguientes requerimientos
pertenecen a este grupo:
\begin{itemize}
\item Definir un nuevo protocolo de comunicación ligero.
\item Mostrar de forma efectiva las formas de publicidad manejadas por la
empresa.
\item Notificar que se realizó la impresión de la creatividad.
\item Notificar que se realizó clic en la creatividad y mostrar el contenido.
\end{itemize}
Los requerimientos Avanzados son aquellos que permiten reutilizar
las publicidades descargadas mientras las mismas estén dentro del
periodo de validez enviado por el servidor. Estos requerimientos ayudan
a resta carga de trabajo al servidor y reducen el tiempo que tarda
en aparecer la publicidad. Este estos requerimientos se encuentran:
\begin{itemize}
\item Definir mecanismo de almacenamiento.
\item Definir mecanismo de rotación.
\end{itemize}
Los requerimientos Adicionales son aquellos que complementan los requerimientos
Básicos y Avanzados para tener un componente que siga las reglas del
negocio de la empresa.
\begin{itemize}
\item Definir mecanismo de colocación de creatividad si el dispositivo no
tiene conexión a red de datos.
\item Reducir el trabajo del servidor a la hora de decidir la creatividad
a enviar.
\item Reporte de la versión del componente y nombre de la aplicación.
\item Permitir la segmentación de los usuarios.
\item Definir mecanismo de segmentación por país y operadora.
\end{itemize}
La lista de estos requerimientos se encuentra contemplada en el Documento
de Requerimientos del Software especificado en el Apéndice \ref{chap:Requerimientos}.Gracias
a esta recaudacion se puede tener una idea general acerca las funcionalidades
del componente sin enfocarse en los detalles de implementación, y
de esta manera tener una mejor idea de qué es lo que se desea implementar.
Esta forma de representación permite conocer lo que el usuario necesita
sin dejar que las decisiones de implementación interfieran.


\subsection{Hardware}

Para el desarrollo del componente en iOS se utilizó como estación
de trabajo una laptop MacBook, con acceso a internet, procesador Intel
Core 2 Duo de 2 GHz, 1 GB de memoria RAM, 160 GB de disco duro y sistema
operativo Mac OS X 10.6.8 Snow Leopard. 

En el caso de la versión de Android, se utilizó una computadora de
escritorio de marca Lenovo, con acceso a internet, procesador Intel
Premium Dual de 2.40 GHz, 2 GB de memoria RAM, 150 GB de disco duro
y sistema operativo de 32 bits Windows XP Professional Version 2002
Service Pack 3. 


\subsection{Software}

El lenguaje utilizado para el desarrollo del componente en iOS fue
Objective-C, usando como IDE (Entorno de Desarrollo Integrado) Xcode
Versión 4.0. 

En el caso del componente en Android, se usó como IDE Eclipse 3.7
Indigo, con un plugin de Herramienta de Desarrollo de Android (Android
Development Tools), y como lenguaje de programación Java. 


\subsection{Usuarios que participan en el sistema}

En el cuadro 5.1 se listan los usuarios que participan en el componente:

\begin{table}[h]
\textbf{\caption{Usuarios que participan en el Sistema}
}

\centering{}%
\begin{longtable}{|c|>{\centering}p{11cm}|}
\hline 
\textbf{Nombre} & \textbf{Descripción}\tabularnewline
\hline 
{\footnotesize Administrador} & {\footnotesize Se encarga de administrar el sistema AdServer MobileX.}\tabularnewline
\hline 
{\footnotesize Desarrollador de Aplicaciones} & {\footnotesize Es el creador de la aplicación donde será desplegada
la publicidad. Entre sus responsabilidades se encuentra el uso apropiado
del componente, para que la publicidad se muestre de la mejor manera
y se haga el reporte de manera apropiada.}\tabularnewline
\hline 
{\footnotesize Desarrollador del la Creatividad} & {\footnotesize Es el creador de la publicidad desplegada en las aplicaciones. }\tabularnewline
\hline 
{\footnotesize Desarrollador del Componente} & {\footnotesize Es el creador del componente para las plataformas móviles
utilizadas en la empresa. Entre sus responsabilidades esta crear un
componente portable, funcional y extensible que sea fácil de utilizar
por el desarrollador de aplicaciones. }\tabularnewline
\hline 
\end{longtable}
\end{table}



\subsection{Riesgos}

Se elaboró una lista de riesgos inicial para la identificación de
riesgos y la mitigación temprana de los mismos, evitando retrasos
en la terminación de los módulos del sistema. Entre los riesgos listados
se encuentran: 
\begin{itemize}
\item Falla Física sobre la Máquina de Trabajo.
\item Falta de Documentación sobre el Protocolo Actual.
\item Falta de Documentación sobre el Componente Actual.
\item Falta de Tiempo del Administrador del Servidor para Realizar Pruebas.
\item Falta de Conocimientos sobre el Software a utilizar
\item Falta de Planificación
\end{itemize}
El plan de riesgos final se encuentra anexado en el Apéndice \ref{chap:Documento-de-Lista}. 


\subsection{Plan de Pruebas}

Se realizó la versión inicial del plan de pruebas. Este documento
incluye una recolección de las pruebas a realizar para la comprobación
del correcto funcionamiento del componente.


\section{Fase de Elaboración}

El principal objetivo de esta fase es asegurar la factibilidad del
componente a desarrollar, a través de un diseño de las funcionalidades
a implementar. 

Las principales actividades realizadas durante esta fase fueron el
diseño de prototipos que cumplieran con los requerimientos iniciales
especificados en la fase anterior (a partir de un modelado por lluvia
de ideas), configuración del ambiente de desarrollo y la refinación
de documentos realizados en la fase de inicio. 

Para la realización de esta fase se realizó una investigación exhaustiva
acerca el sistema MobileX, tanto del servidor como del cliente, con
el objetivo de tener una idea acerca su funcionamiento, y las mejoras
necesarias correspondientes al modelo de negocio de la empresa. Esta
fase fue planificada para realizarse en una iteración. 


\subsection{Modelado por Lluvia de Ideas}

El modelado por lluvia de ideas es un método que comprende la presentación
de un conjunto de ideas, que luego son analizadas para concluir la
mejor solución de acuerdo a los requerimientos, procediendo a un refinamiento
de casos de uso. Esta técnica se llevó a cabo para cada uno de los
requerimientos iniciales: 


\subsubsection{Definir un nuevo protocolo de comunicación ligero}

Para la comuniación entre el AdServer MobileX y el componente AdField
MobileX se usa XML-RPC. El componente envía un XML con el número de
zona y una serie de datos vacios, y recibe un xml, con muchos datos
que no utiliza. Como las aplicaciones en los dispositivos móviles
se ven afectadas por la cantidad de datos que envían y reciben, el
objetivo del nuevo protocolo es que sea lo más simple y reducido que
se pueda.


\subsubsection*{Mantener la comunicación con XML-RPC y reducir los datos enviados}

Al evaluar la estructura XML que envía el componente al solicitar
la publicidad vimos que actualmente la unica información importante
es el número de zona. Los demás datos son vacíos y el servidor no
hace uso de ellos. Así que se podría enviar el XML con ese único valor.

Por otra parte, después de evaluar la estructura XML-RPC recibida,
donde solo una pequeña parte servía para la impresión y el reporte
de las creatividades, definimos una nueva respuesta que enviaría el
servidor. La estructura nueva enviará unicamente los siguientes campos: 
\begin{enumerate}
\item Dirección url donde se encuentra la creatividad.
\item Dirección url que será llamada cuando se muestre la creatividad en
la aplicación.
\item Dirección url que será llamada cuando se haga clic sobre la creatividad
en la aplicación.
\item Ancho que mide la creatividad.
\item Alto que mide la creatividad.
\item Identificador identificador del tipo de creatividad.
\item Tipo de contenido.
\item Tipo de campaña.
\item Fecha de vencimiento.
\item Tiempo en cache.
\end{enumerate}

\subsubsection*{Realizar la comunicacion mediante JSON-RPC y reducir los datos enviados}

A pesar de que la cantidad de datos enviados por ambas partes se reduce
con la solución propuesta anteriormente, JSON-RPC ofrece una comunicación
mucho más ligera, por usar el formato JSON para codificar las estructuras.
Por lo tanto esta solución es mucho mejor que la anterior. 


\subsubsection*{Realizar la comunicacion mediante un híbrido del JSON-RPC}

El AdServer MobileX es un servidor que realiza diversos servicios,
por lo que se busca también quitarle un poco de carga. Enviar unos
cuantos argumentos en un JSON generará que el servidor deba parsear
esta estructura para obtener los datos que necesita para responder
con la creatividad adecuada. Por lo tanto, después de varias discusiones
con miembros de la empresa encargados del servidor, se decidió colocar
los argumentos en la dirección url a la cual se realiza la peticion
de publicidad. Por lo tanto, si se desea enviar un numero de zona,
el url lucirá algo como:

\begin{center}
http://direccion.del.servidor/procedimientoremoto.php?zoneid=1
\par\end{center}

Se decidió que la respuesta del servidor fuera un JSON como se planteó
anteriormente.

Sin embargo, las plataformas iOS y Android no tienen una librería
nativa para la decodificación de archivos JSON en las versiones mínimas
que se desean soportar (estas versiones son la 3.2 para Android y
la 3.0 para iOS). Por esta razón se deben buscar librerías de terceros
que perminan hacer la decodificación.


\subsubsection{Reducir el trabajo del servidor a la hora de decidir la creatividad
a enviar}

Por cada campaña, el servidor mantiene diversos tamaños de una misma
creatividad, a fin de enviar al cliente el más apropiado para el dispositivo
desde el cual está realizando la petición. Para tomar la decisión
de que tamaño enviar, el servidor obtiene de los parámetros provenientes
de la solicitud http el agente de usuario (user agent), posteriormente
lo busca en una tabla, para obtener los tamaños de la pantalla soportados
por el dispositivo. Gracias a los tamaños, busca una creatividad que
sea menor al área de la pantalla del dispositivo. Si no encuentra
el agente de usuario en la lista, envia una creatividad por defecto,
la cual puede no ser la más apropiada para el dispositivo.

La tabla con la lista de los agentes de usuario es llenada a mano
por el administrador del servidor. Esta lista debe ser actualizada
cada vez que sale un nuevo sistema operativo o un navegador (o una
nueva versión de los que ya existen).

La solución a este problema es que se envíe en los argumentos de la
petición de publicidad el alto y el ancho de la pantalla del dispositvo,
de manera de quitarle al servidor el paso de buscar en la lista. Por
lo tanto, el url lucirá como: 

\begin{center}
http://direccion.del.servidor/procedimientoremoto.php?zoneid=1\&width=1\&height=1
\par\end{center}


\subsubsection{Reporte de la versión del componente y nombre de la aplicación}

Para los datos estadísticos que maneja el servidor con respecto a
la utilización del componente por parte de las diferentes plataformas
y aplicaciones, se desea que el servidor reciba por parte del cliente
el número de la versión del componente y el nombre de la aplicación
que lo está usando. Después de discutir con otros miembros de la empresa
especializados en el servidor, se decidió que la mejor manera de enviar
estos datos es agregarlos al agente de usuario que se envía al servidor.
Por lo tanto, un agente de usuario lucirá de la sigueinte manera:

\begin{center}
<Agente\_de\_Usuario> \textbf{MobileX/<Número\_Versión>} \textbf{<Nombre\_Aplicación>}
\par\end{center}


\subsubsection{Permitir la segmentación de los usuarios}

El sistema AdServer MobileX permite el registro de usuarios a través
de las aplicaciones para disfrutar de algunos servicios que ofrece.
Al hacer este registro el servidor envía al cliente una cadena de
caracteres que sirve para realizar la segmentación de los usuarios.
Se desea que si se posee esta cookie, se pueda enviar al servidor,
de manera de que pueda enviar inteligentemente la publicidad, es decir,
si en una zona se tiene una publicidad de una zapatería para niños
y otra de una tienda de deportes, se envíe la primera si la persona
es una mujer entre los 30 y los 40 años, mientras que es mejor enviar
la segunda si la persona es un hombre entre los 18 y los 60 años.

Se decidió enviar este valor en la solicitud que se hace al servidor
como un argumento más. Por lo tanto, el url lucirá como:

\begin{center}
http://direccion.del.servidor/procedimientoremoto.php?zoneid=1\&cookie=C00K13
\par\end{center}


\subsubsection{Definir mecanismo de segmentación por país y operadora\label{mcc-mnc}}

Mobile Media Netowrks ofrece sus servicios a diferentes países de
America Latina, por lo que es necesario lograr segmentar la entrega
de las campañas por países, o por operadoras. Por ejemplo, una empresa
de telefonía presente en varios países podría colocar una creatividad
anunciando los nuevos precios de un determinado plan de datos. Como
la moneda varía según el país, se desea que aparezca la creatividad
con la moneda apropiada según la ubicación. Además, es probable que
la empresa desee que esta campaña solo aparezca en dispositivos que
pertenecen a esta misma operadora. 

Una de las soluciones planteadas fue utilizar las cookies recibidas
al realizar el registro, puesto que uno de los datos de la cookie
es el país, sin embargo, esto no es viable porque muchas de las aplicaciones
de móviles no tienen un mecanismo de registro.

Otra de las posibles soluciones es utilizar las librerías nativas
de los dispositivos para obtener el MCC y el MNC. El Código de País
del Dispositivo (Mobile Country Code, o MCC) es un número único que
se asigna a cada país. Por ejemplo, si el usuario está en Venezuela,
su MCC es 734. Por otra parte, el Código de Red del Dispositivo (Mobile
Network Code, o MNC) es un número único que se asigna a una operadora
en un determinado país. Sin el MCC, no es posible saber cual operadora
es la que se está consultando. Envíando el MCC y el MNC como argumentos
tanto en la petición de publicidad como en el reporte de la impresión
y el clic, el servidor puede generar estadísticas y reportes interesantes. 

Sin embargo, en varias discusiones con otros miembros de la empresa
relacionados a la administración del servidor y la colocación de la
publicidad, se decidió también enviar el número ip del dispositivo.
Esta decisión está derivada de que los portales web no pueden obtener
el mcc, ni el mnc, por lo que se necesita que ellos también puedan
enviar datos sobre su ubicación. La ventaja de usar el ip es que todos
los dispositivos y portales web pueden enviar este dato, la desventaja
es consultar el país a partir de un ip significa consulta una tabla
de millones de filas. Por lo tanto, el servidor usará primero el MCC
y el MNC si estos valores no son vacíos. 

De esta manera, el url de solicitud lucirá como:

\begin{center}
http://direccion.del.servidor/procedimientoremoto.php?zoneid=1mcc=734\&mnc=4\&ip=192.168.0.0
\par\end{center}


\subsubsection{Mostrar de forma efectiva las formas de publicidad manejadas por
la empresa}

Mobile Media Networks maneja actualmente solo dos formatos de creatividades,
estos son jpg y gif. En cualquier plataforma se pueden mostrar las
imagenes planas sin ningún inconveniente (este es el caso de las imagenes
jpg). Sin embargo, las plataformas iOS y Android no tienen una librería
nativa para mostrar contenido animado (contenido gif) en las versiones
mínimas que se desean soportar.Por lo tanto, se deben buscar librerías
de terceros que permitan realizar la animación de las imagenes.


\subsubsection{Notificar que se realizó la impresión de la creatividad}

En el componente AdField, el reporte de la impresión de la creatividad
se realiza haciendo una petición HTTP con la dirección url que envía
el AdServer. Esta dirección es generada por el servidor, por lo que
contiene los argumentos necesarios para hacer el reporte correcto.
Si se le agregan argumentos extra, el servidor los ignora. Crear una
nueva dirección a mano sería ineficiente.Por lo tanto, se usará la
misma forma de reporte que el AdField.


\subsubsection{Notificar que se realizó clic en la creatividad y mostrar el contenido}

Al igual que la dirección de reporte, la dirección de clic es enviada
por el servidor conteniendo todos los argumentos necesarios para mostrar
el contenido del portal web asociado a la creatividad, así como la
realización del reporte. Cuando se hace clic a la publicidad en una
aplicación, la misma debería abrir un navegador web con la información
detallada de lo que se quiere mostrar. Por ejemplo, si la creatividad
es de un banco, al darle clic posiblemente lleve al portal web del
mismo banco. El reporte se hace de la misma forma que en el caso de
una impresión, es decir, mediante una petición HTTP.

En las investigaciones hechas de las dos plataformas, se descubrió
que ambas tienen una librería para poder mostrar en un navegador el
contenido de una dirección url. Internamente la librería debe hacer
una petición HTTP, así que no se requiere de ningún mecanismo extra
para el reporte.


\subsubsection{Definir mecanismo de almacenamiento}

Tener un mecanismo de almacenamiento de las creatividades y sus datos
asociados permite que la obtención de la publicidad sea más rápida,
ya que se estaría sacando del mismo dispositivo, en lugar de depender
de la conexión de datos y su velocidad. Ambas plataformas permiten
almacenar elementos en el directorio propio de la aplicación, en un
directorio cache, o mediante otros mecanismos propios como bases de
datos o listas de preferencias. 

Viendo las ventajas y desventajas de cada uno se decidió que las creatividades
se guarden en el directorio cache. La ventaja es la rapidez de carga
que se tendrá de la creatividad. La desventaja es que el directorio
cache de la aplicación podría ser borrado por decisión del sistema
operativo; sin embargo, esto no sería un problema, ya que la creatividad
no es información sensible para la aplicación. 

Con respecto al almacenamiento de las características de las creatividades
almacenadas, se decidió tener una lista de preferencias. Ambas plataformas
ofrecen librerías para el acceso y escritura de los datos, donde el
usuario no tiene que manejar la estructura verdadera. Además, son
apropiadas para mantener pocos datos almacenados. 


\subsubsection{Definir mecanismo de rotación\label{rotacion}}

Debido a que las creatividades y su información relacionada se desean
almacenar en el dispositivo para quitarle carga al servidor y reducir
el tiempo de espera por la publicidad, se necesita de un mecanismo
que establezca cuando se debe pedir una nueva creatividad. El servidor
enviará la fecha de expiración de la creatividad, cada vez que se
va a mostrar, se chequea esta fecha. Si venció, la creatividad se
descarta y se pide otra. En caso contrario se considera a la creatividad
como vigente y se mostrará. 

Además del tiempo de expiración, el servidor también envía el tiempo
en cache, el cual determina el tiempo que debe usarse una creatividad
antes de pedir otra nueva al servidor. Esto se hace porque la mayoría
de las campañas duran más de una semana sin cambiar de creatividad,
por lo que muchas veces se piden creatividades que ya se tienen almacenadas.
Este tiempo le indica al cliente que no haga otra solicitud hasta
que pase el tiempo enviado. 

Cuando se pida una creatividad de cache, se revisará primero si es
vigente, es decir, si no ha expirado. Si es vigente, se revisará si
es válida. Ser válida significa que, desde el momento que se descargó,
no se ha pasado el tiempo que envío el cache de espera (tiempo de
cache). Por lo tanto, si la creatividad es válida, se mostrará, pero
no se pedirán nuevas creatividades al servidor. De no ser válida (pero
si vigente) se mostrará la creatividad pero se pedirá otra al servidor.
Esta nueva creatividad se almacenará en cache como todas las anteriores,
y se colocará al final de una cola creada con las listas de preferencias.
La próxima vez que se ingrese a la aplicación, la creatividad que
debe aparecer es la última descargada.

\begin{figure}[h]
\centering{}%%\includegraphics[scale=0.6]{Images/rotacion}\caption{Política de rotación de banners}
\end{figure}



\subsubsection{Definir mecanismo de colocación de creatividad si el dispositivo
no tiene conexión a red de datos}

Uno de los grandes problemas cuando los dispositivos no poseen red
de datos es que en la aplicación queda un espacio inutilizado porque
la creatividad no se muestra. Se puede aprovechar el hecho de tener
un mecanismo de almancenamiento para evitar que esto suceda.


\subsection{Refinamiento del Plan de Pruebas}

En esta fase se realizó un refinamiento al plan de pruebas adaptado
a las estructuras definidas en base a las secciones anteriores. El
plan de pruebas puede ser visualizado en el apéndice F.


\subsection{Riesgos Técnicos}

Al finalizar esta fase se determinaron los riesgos técnicos relacionados
con el diseño proveniente del modelado por lluvia de ideas:
\begin{enumerate}
\item Soporte del componente de nuevas formas publicitarias. Por ejemplo:
Rich Media.
\item Soporte del componente de nuevas necesidades del mercado.
\end{enumerate}

\section{Fase de Construcción}

El objetivo de la fase de construcción es desarrollar los componentes
publicitarios hasta el punto que se encuentre listo para la pre-producción
de pruebas. El énfasis en las fases anteriores fue priorizar y comprender
los requerimientos, utilizando el modelado para atacar una solución.
En esta fase se recurre a la codificación, presentando primeras versiones
del componente que muestra la publicidad en las aplicaciones. 

Esta fase fue realizada en tres iteraciones por cada plataforma desarrollada. 

A continuación se hace una descripción de las decisiones tomadas y
los procedimientos establecidos, haciendo una distinción en el caso
que amerite por plataforma, ya que en general el componente se comporta
de igual manera en ambas plataformas..


\subsection{Primera Iteración}


\subsubsection{Diseño e implementación de un mecanismo para mostrar las creatividades}

Antes de específicar como es el proceso de muestra de creatividades
propio por plataforma, se debe mencionar que las interfaces de usuario
no pueden tener componentes que se bloqueen por más de cierto tiempo,
porque entorpecen el funcionamiento de la aplicación o causan que
la aplicación se cierre forzosamente. Por lo tanto es necesario que
desde el comienzo desde la primera llamada al componente, se cree
un nuevo hilo de ejecución, que permita realizar toda la lógica del
componente sin causar ningún efecto desfavorable a la aplicación principal\@.


\subsubsection*{iOS}

Una creatividad podría catalogar en iOS como una Vista de Imagen (Image
View) sin embargo, el componente tendrá muchas más funcionalidades
que una imagen normal. Investigando arduamente las formas de agregar
componentes personalizados a la interfaz de usuario, se descubrió
que la manera más sencilla es la creación de un delegado (Delegate)
y controlador (Controller). 

El programa delegado debe ser una extensión de una Vista Imagen, para
luego poder asociar la Vista Imagen que se coloque en la interfaz
de usuario a la nueva que se está definiendo. El delegado se encargará
de realizar los siguientes pasos:
\begin{enumerate}
\item Agregar a la dirección url donde se solicita la publicidad los argumentos
obligatorios y opcionales. El único argumento obligatorio es el número
de zona. Los demás argumentos definidos en la lista de requerimientos
serán opcionales.
\item Establecer una conexión HTTP con el servidor.
\item Obtener la respuesta del servidor, la cual está en formato JSON y
aplicarle un decodificador.
\item Tomar los campos obtenidos con el decodificador y enviarlo al controlador.
\end{enumerate}
Es importante resaltar que la versión de iOS que se quiere soportar
no tiene un codificador/decodificador nativo de JSON, por lo que se
usó la librería de terceros JSONKit. JSONKit está licenciado bajo
BSD y Apache 2.0.\cite{Cap5.JSONKit}

El programa controlador es el que se encargará de modificar la interfaz
de usuario. A partir de los argumentos recibidos desde el programa
delegado, deberá descargar la creatividad y colocarla de acuerdo a
las dimensiones originales. Dado que las dos formas de creatividad
que maneja actualmente la compañía son las imagenes en formato jpg
y gif, el controlador debe colocar cualquiera de las dos sin problemas.
La versión de iOS que se quiere soportar tampoco tiene una librería
nativa para el manejo de imágenes tipo gif, por lo que se necesitó
de una libreria de terceros, llamada Animated Gif \cite{Cap5.AnimatedGif},
para poder colocarlos. Esta librería coloca cuadro por cuadro del
gif, sin dejar el primero cuadro de base, como hacen otras librerías. 


\subsubsection*{Android}

Como se estableció anteriormente, el componente publicitario tiene
una lógica oculta al usuario donde se implementan más funcionalidades
que las obvias. Por lo tanto, aunque una creatividad podría catalogar
como una Vista Imagen (Image View) en Android. Luego de realizar una
investigación exhaustiva sobre las formas de agregar componentes personalizados
a la interfaz de usuario, se descubrió que, en el caso de Android,
la manera más simple es la creación de un componente de interfaz y
un programa controlador.

Un componente de interfaz en Android es codigo XML que establece las
características de un componente. Cuando se tienen componentes personalizados,
queda a cargo del programa controlador definir cual será el comportamiento
del mismo. Para establecer cual será el controlador del componente
personalizado, el nombre del mismo debe ser la ruta completa del nombre
del programa que será el controlador.

El programa controlador debe ser una extensión de una Vista, para
poder realizar las modificaciones al componente personalizado. El
programa controlador realizará las mismas ladores que el programa
delegado de iOS definido en esta misma sección.

Como el programa controlador se encargará de modificar la interfaz
de usuario, debe ser capaz de mostrar en el mismo espacio cualquiera
de las formas de publicidad manejadas por la empresa. Para la versión
de Android que se quiere soportar no existe una librería nativa para
el manejo de imágenes tipo gif, por lo que se necesitó de una libreria
de terceros, llamada GifView \cite{Cap5.GifView}. Esta libreria coloca
el primer cuadro fijo y coloca uno por uno los demás sobre este.


\subsubsection{Diseño e implementación de un mecanismo para el reporte de la impresión}

Tanto para iOS como para Android, el programa controlador debe ser
el encargado de llamar a una función que realice una petición HTTP
al servidor cuando la creatividad sea descargada y mostrada en la
interfaz de usuario. Como la dirección url ya trae los argumentos
que necesita el servidor para hacer el reporte correcto, solo se debe
hacer la llamada.


\subsubsection{Diseño e implementación de un mecanismo para el reporte del clic
y la colocación apropiada del contenido}


\subsubsection*{iOS}

Con respecto al reporte del clic, este procedimiento es más delicado,
no solo sirve realizar una petición HTTP, puesto que la dirección
de clic trae como argumento la dirección del portal web al cual se
redireccionará el navegador.

Leyendo la documentación, no existe para la versión de iOS que se
quiere soportar un componente que permita abrir un explorador dentro
de la aplicación. Sin embargo, si se puede cargar la información en
una Vista Web sencilla dentro de la aplicación o lanzar el explorador
prederteminado. La ventaja de la primera opción es que no se abandona
la aplicación al hacer clic sobre la creatividad, pero la Vista Web
no provee las interfaz de un navegador. La ventaja del explorador
es que se tienen todas las funciones de navegador que permiten al
usuario disfrutar de una buena interfaz, pero el navegador se abre
fuera de la aplicación. Se decidió utilizar la segunda opción. 

Al lanzar un navegador con una dirección url, el mismo realizará la
llamada HTTP, por lo que se estará reportando el clic automáticamente.


\subsubsection*{Android}

Al igual que en iOS, no existe un componente en Android que permita
abrir un navegador dentro de la aplicación. Sin embargo, por la estructura
de Android, no es necesario. Android maneja cada interfaz por medio
de Actividades, cuando se quiere lanzar un navegador web, se crea
una nueva actividad, que se empila en la lista de actividades. Si
la persona decide salir del navegador, entonces este se quitará de
la pila y la persona regresará a la pantalla donde estaba la aplicación
antes de hacer clic en el banner. 

A pesar de esta ventaja, se investigaron otras formas de colocar contenido
web dentro de la aplicación. Al igual que iOS, Android tiene una Vista
Web, la cual permite cargar de manera sencilla contenido web. Sin
embargo, no se disfruta de la interfaz propia de un navegador y la
implementación es muy básica así que el contenido no se ve atractivo.

Por lo tanto, se decidió utilizar la primera opción, lanzar un navegador.
Al hacer esto, se realizará la llamada HTTP, y por lo tanto, se reportará
el clic automáticamente.


\subsection{Segunda Iteración}


\subsubsection{Diseño e implementación de un mecanismo de almacenamiento}

El objetivo que se busca lograr al almacenar la publicidad en el dispositivo
es reducir la cantidad de peticiones que se hacen al servidor. Además,
se busca que el tiempo que transcurre entre que la aplicación se abre
y la creatividad se muestre sea el mínimo. 

Almacenar la publicidad implica no solo descargar la creatividad y
almacenarla, sino también todos los datos asociados a ella. Por lo
tanto se debe definir un mecanismo para el almacenamiento de la imagen
y otro para los datos. 


\subsubsection*{iOS}

Para el caso del almacenamiento de las creatividades, se usará el
directorio Cache de la aplicación, lo que permite un acceso rápido
a los datos. El nombre del archivo corresponderá con identificador
de la creatividad (bannerID) para poder acceder a él posteriormente.

Con respecto a las características, necesitamos un mecanismo que permita
almacenar arreglos, diccionarios y tipos de datos básicos. Las opciones
que proporciona iOS para la persistencia de datos son:
\begin{enumerate}
\item \textit{NSUserDefaults}: Están basados en las listas de preferencias.
Permite almacenar información sencilla, como las prefencias de los
usuarios de una aplicación. 
\item \textit{Plist}: Permite almacenar datos en una lista de preferencias.
El acceso es más rápido que en las Preferencias de Usuarios (NSUserDefaults). 
\item \textit{SQLite}: Es un archivo loca que actía como servidor de base
de datos. No es útil si se quieren hacer muchos accesos concurrentes.
\item \textit{Core Data}: Permite almacenar datos un poco más complejos.
Usa SQLite como manejador de base de datos por defecto. 
\end{enumerate}
Los datos de la creatividad no son sensibles, es decir, la aplicación
seguirá ejecutandose normalmente si se borran y no tienen información
relevante sobre el usuario. De las cuatro opciones anteriores, la
más cercana a las necesidades en la opción dos, utilizar una plist.
Las listas de preferencias de iOS permiten almacenar tipos básicos
de datos, así como arreglos y diccionarios.


\subsubsection*{Android}

Al igual que iOS, Android permite el almacenamiento de archivos en
el directorio Cache de la aplicación, permitiendo un acceso rápido
a los datos. De igual manera se usará el identificador de la creatividad
(bannerID) como nombre del archivo.

Las opciones del almacenamiento que ofrece Android son las siguientes:
\begin{enumerate}
\item \textit{Shared Preferences}: Permite almacenar datos privados de los
tipos primitivos de la forma clave-valor.
\item \textit{Internal Storage}: Permite almacenar datos privados en la
memoria del dispositivo.
\item \textit{External Storage}: Permite almacenar datos en memorias externas.
\item \textit{SQLite Databases}: Permite almacenar datos en una base de
datos privada.
\item \textit{Network Connection}: Permite almacenar datos en un servidor
propio.
\end{enumerate}
La opcion número uno es la que más se acerca a las necesidades del
almacenamiento. El único problema es que solo permite almacenar los
tipos de datos primitivos, es decir, no se pueden crear diccionarios
ni arreglos, por lo que se debe disponer de más de un archivo de preferencias. 

Por lo tanto, para una zona existirá un archivo principal, donde se
guardarán los números identificadores de las creatividades descargadas,
y un archivo adicional con las características y datos importantes
de la creatividad en específico.


\subsubsection{Diseño e implementación de un mecanismo de rotación}

Tener un mecanismo de rotación de publicidad permite, en conjunto
con el mecanismo de almacenamiento, reducir la cantidad de peticiones
al servidor y reducir los tiempos de carga. No se debe confundir el
termino rotación con el hecho de cambiar las creatividades en una
pantalla cada cierto tiempo. Esto no fue definido como requerimiento,
por lo tanto, no fue implementado. 

La rotación de las creatividades debe hacerse en los archivos donde
se mantiene la lógica de la persistencia. Es en estos archivos, mencionados
en el punto anterior, donde están los datos que permiten saber cuales
son las creatividades válidas. 

Al momento de descargar una nueva publicidad, esta se debe colocar
al final de la lista de creatividades, y si se llegó al máximo, borrar
la primera de la lista y correr todas una posición. A pesar de que
suena a que hay muchas creatividades almacenadas, realmente una zona
tendrá máximo 2, la que se mostrará al abrir la aplicación y la que
se descargará para ser mostrada la siguiente vez que se abra la aplicación.
Además de mantener las listas en orden, uno de los datos que se deben
almacenar de la creatividad y que es recibido por el servidor es el
tiempo de descarga. Este tiempo, en conjunto con el Tiempo de Cache
y la Fecha de Vencimiento que envía el servidor, permite saber si
una creatividad es válida, vigente o si venció, como se explicó en
la sección \ref{rotacion}.

En el momento en que una creatividad esté vencida, o sea eliminada
de la lista, se borrarán todos los datos y archivos relacionados a
ella. Si la creatividad que se descarga nueva ya está en la lista,
solo se colocará al final, pero no se descargarán las imágenes de
nuevo.


\subsection{Tercera Iteración}


\subsubsection{Diseño e implementación de un mecanismo para el comportamiento cuando
no se tiene red de datos}

Existe la posibilidad de que un dispositivo móvil no tenga acceso
a una red de datos en un momento en específico, pero esto no necesariamente
causa que el usuario no pueda usar una aplicación. Muchas aplicaciones
pueden ser usadas sin necesidad de tener internet. Sin embargo, cuando
se diseña una aplicacion, se asigna un espacio de la interfaz de usuario
para mostrar el contenido de la publicidad, y sin red de datos este
espacio queda inutilizado. 

Aprovechando el mecanismo de almacenamiento, se puede disponer de
publicidad aunque no se tenga red de datos. Mientras que la creatividad
que está almacenada no venza, podrá mostrarse, pero la funcionalidad
del clic no servirá puesto que no hay internet.

Se consideró el caso de querer conocer la cantidad de impresiones
que se hacen aunque el dispositivo este sin red, por lo que se almacenará
una nueva característica a la información almacenada. Una vez que
se tenga acceso a la red, se enviará al servidor este dato. Esta característica
es opcional, ya que hay aplicaciones que no funcionan sin red de datos
y la publicidad que se mostró no es de relevancia en los reportes.


\subsubsection*{iOS}

A pesar de que Android tiene una librería para saber si en un determinado
momento el dispositivo tiene conexión, iOS no cuenta con una para
las versiones que se necesitan soportar. Para poder conocer el estado
de la red, se necesitó utilizar un componente que no se encuentra
en las librerías de iOS llamada Reachability, la cual es distribuida
por Apple Inc.


\subsubsection{Diseño e implementación de un mecanismo para la segmentación}

El servidor ofrece como uno de sus servicios la posibilidad de que
los usuarios se regristren. Al realizar el registo, el servidor retorna
una cadena de caracteres o galleta (cookie), que contiene los datos
de registro de una manera sencilla y corta. 

El componente debe permitir agregar esta galleta a la petición que
se le hará al servidor. Si el servidor recibe la galleta, podrá enviar
una publicidad más apropiada al usuario. La solución propuesta es
usar la lista de preferencias (plist) para almacenar este valor en
el caso de iOS, y una lista de preferencias compartidas para Android.
La galleta se almacenará bajo la etiqueta \textit{cookie, }permitiendo
que el componente envíe la galleta que tenga almacenada en cada solicitud.

Si no se tiene almacenada una galleta, el argumento en la solicitud
será vacío, y por lo tanto el servidor lo ignorará. Si los registros
cambian con el tiempo y el formato de la galleta también, solo se
tendrá que sustituir el valor anterior por el nuevo. 

Otro servicio que ofrece Mobile Media es enviar publicidad especifica
unicamente a los dispositivos o portales que se encuentren en un país
o que pertenezcan a una operadora. Para poder hacer esto, el componente
debe enviar algún dato que permita conocer estos valores. Como se
discutió en la sección \ref{mcc-mnc}, estos valores son el Código
de País del Dispositivo (MCC), el Código de la Operadora del Dispositivo
(MNC) y la dirección IP. Gracias a las librerías propias de las plataformas,
estos datos se pueden obtener fácilmente.

Por último, también es necesario conocer el dispositivo que realiza
la solicitud, ya que el servidor ofrece el servicio de enviar una
publicidad a un dispositivo en específico. Gracias al agente de usuario,
podemos conocer cual es la marca y modelo del dispositivo. En un futuro,
el agente de usuario también servirá para conocer si un dispositivo
soporta formas de publicidad nuevas. 

Además del agente de usuario se busca incluir también la versión del
componente y el nombre de la aplicación, lo que generará estadísticas
más robustas y la ventaja de enviar publicidad dirigida específicamente
a los usuarios de cierta aplicación. Este mecanismo fue muy fácil
de implementar, ya que tanto iOS como Android ofrecen librerías que
permiten obtener estos valores rápidamente.


\section{Fase de Transición}

En esta última fase se realizaron las pruebas de revisión de ambos
componentes a partir del plan de pruebas creado, dando como resultado
un prototipo 100\% funcional para el componente AdView MobileX de
Android y un prototipo 100\% funcional para el componente AdViewMobileX
de iOS. 

Por otra parte, se finalizó la documentación general de los componentes
y el refinamiento de los documentos. También se realizaron los documentos
de utilización del componente, los cuales se encuentran en los apéndices
y .Así mismo se realizaron las presentaciones finales del componente
a los involucrados con el mismo, entregando los artefactos que lo
comprenden y toda la documentación descrita en este informe. 

\newpage{}



\chapter*{Conclusiones y Recomendaciones}

\selectlanguage{english}%
\addcontentsline{toc}{chapter}{Conclusiones y Recomendaciones}

\selectlanguage{spanish}%
\vspace{5mm}


La empresa Mobile Media Networks utiliza el sistema AdServer MobileX
como una plataforma publicitaria para la distribución de publicidad
en portales web y dispositvos móviles. Para la colocación simple y
apropiada en las aplicaciones, el AdServer MobileX se comunica con
un componente publicitario llamada AdView MobileX (o AdField, para
la versión de Blackberry). Este componente hace transparente al usuario
toda la lógica de negocios que está implementada, ofreciendo al AdServer
datos importantes para la segmentación y el reporte, además de reducir
la carga del mismo.

El presente proyecto consistió en el diseño de un protocolo de comunicación
simple y compacto, y el diseño y la implementación de un componente
que cumple con los requerimientos establecidos al comienzo de la pasantía.
Estos requerimientos se dividieron en tres categorías: Requerimientos
Básicos, Requerimientos Avanzados y Requerimientos Adicionales.

A partir de los requerimientos básicos se implementó un componente
que tiene la capacidad de construir una petición a un servidor, realizar
la comunicación e interpretar la respuesta. A partir de esta respuesta
coloca la creatividad en el lugar y de la manera adecuada, soportando
todas las formas publicitarias que maneja la empresa y sin afectar
de ninguna manera la interfaz de usuario.

Los requerimientos avanzados aportaron al componente un poco más de
inteligencia, siendo capaz de decidir cuando pedir una publicidad
y cuando aprovechar lo máximo una que ya se descargó. Además, las
creatividades se cargan inmediatamente al momento de abrir la aplicación,
haciendo que el componente no deje espacios inútiles en la interfaz.

Los últimos requerimientos aportaron al componente la posibilidad
de contribuir con información para las estadísticas y reportes que
genera el servidor, además de aportar la lógica necesaria para mostrar
la publicidad a pesar de no tener conexión de red.

La puesta en producción de los módulos del servidor que permitan realizar
la comunicación con el nuevo protocolo, o que permitan obtener nueva
información que envía el componente no están en el alcance del proyecto.
Esto obedece a una decisión que debe tomar la gerencia de la empresa
en conjunto con los miembros de cargos más altos.

La metodología AUP contribuyó con la planificación del proyecto y
a la presentación de la información, la cual servirá a futuros desarrolladores
que se involucren con los dos componentes implementados.

La experiencia obtenida durante el desarrollo de este proyecto de
pasantía comprende nuevos elementos de aprendizaje vinculados tanto
a las nuevas tecnologías utilizadas como al desenvolvimiento laboral
y las nuevas relaciones construidas en la empresa.

Para terminar, se presentan algunas recomendaciones:
\begin{itemize}
\item Incorporación de funcionalidades para poder colocar elementos de contenido
interactivo Rich Media.
\item Incorporación de una funcionalidad que consista en que cuando el usuario
haga clic en una creatividad que promociona un servicio de otra aplicación,
chequee si la apliación existe, y de existir, lance la aplicación
en la sección donde se encuentra la venta del servicio.
\item Realizar la documentación de los nuevos protocolos y componentes,
pero tanbién de los anteriores y de las nuevas funcionalidades que
se incorporarán al servidor. El sistema MobileX en conjunto es muy
complejo, con clientes publicitarios escritos para distintas plataformas
y un servidor muy amplio con muchos servicios ofrecidos.
\end{itemize}
\begin{onehalfspace}
\newpage{}\end{onehalfspace}


\selectlanguage{english}%
\setlength{\parskip}{7pt}

\selectlanguage{spanish}%
\renewcommand{\bibname}{Referencias Bibliográficas}
\begin{thebibliography}{10}
\bibitem{Cap1.Hitos}Hitos y trayectoria: Qué hemos conseguido hasta
hoy. Disponible en http://www.adsmediamobile.com/es/compania/hitos-y-trayectoria,
consultado el 19 de noviembre de 2012.

\bibitem{Cap1.AdsMedia}AdsMedia Mobile Advertising. \textquotedblleft{}Misión
y visión\textquotedblright{}. Presentación interna.

\bibitem{Cap1.Mobile}Mobile Media Networks. \textquotedblleft{}Pasante
de Tecnología\textquotedblright{}. Presentación interna.

\bibitem{Cap2.ClienteServidor}Client Server Networks. http://compnetworking.about.com/od/basicnetworkingfaqs/a/\newline
client-server.htm, consultado el 11 de agosto de 2012.

\bibitem{Cap2.Useragent}Agente de Ususario. http://whatsmyuseragent.com/,
consultado el 25 de diciembre de 2012.

\bibitem{Cap3.AUP1}Metodología AUP (Agile Unified Process \textendash{}
Proceso Unificado Ágil) para Implementación del ERP ADempiere \textendash{}
Parte 1. http://ubuntu-adempiere.blogspot.com/2011/09/metodologia-aup-agile-unified-process.html

\bibitem{Cap3.AUP2}The Agile Unified Process (AUP), http://www.ambysoft.com/unifiedprocess/agileUP.html,
consultado el 16 de agosto de 2012.

\bibitem{Cap3.AUPFases}Fases del AUP. http://cgi.una.ac.cr/AUP/html/phases.html,
consultado el 10 de agosto de 2012.

\bibitem{Cap4.Android}Android. http://www.android.com/about/, consultado
el 21 de noviembre de 2012.

%%\bibitem{key-2}About the iOS Technologies. http://developer.apple.com/library/ios/\#documentation/\newline
%%miscellaneous/conceptual/iphoneostechoverview/Introduction/Introduction.html,
%%consultado el 21 de noviembre de 2012.

\bibitem{Cap4.JSON}JSON http://www.json.org/json-es.html, consultado
el 21 de noviembre de 2012.

\bibitem{Cap4.RPC1}A.Tanenbaum. Redes de Computadoreas. 4ta edición.
Prentice Hall, 2003.

\bibitem{Cap4.RPC2}W.Stallings. Sistemas Operativos. Aspectos Internos
y Principios de Diseño. 5ta edición. Pearson 2005.

\bibitem{Cap4.XMLRPC}http://xmlrpc.scripting.com/, consultado el
21 de noviembre de 2012.

\bibitem{Cap4.JSONRPC}http://json-rpc.org/, consultado el 21 de noviembre
de 2012.

\bibitem{Cap5.JSONKit}JSONKit. https://github.com/johnezang/JSONKit,
consultado el 27 de diciembre de 2012.

\bibitem{Cap5.AnimatedGif}AnimatedGif. https://github.com/scspijker/iOS\_AnimatedGif,
consultado el 27 de diciembre de 2012.

\bibitem{Cap5.GifView}GifView. http://code.google.com/p/android-gifview/,
consultado el 27 de diciembre de 2012.

\bibitem{Cap5.DataStorage}Data Storage. http://developer.android.com/guide/topics/data/data-storage.html,
consultado el 29 de diciembre de 2012.

\bibitem{Cap5.DataPersistence}Data Persistence. http://mobile.tutsplus.com/tutorials/iphone/iphone-sdk\_store-data/,
consultado el 29 de diciembre de 2012.

\end{thebibliography}


\selectlanguage{english}%
\setlength{\parskip}{7.5pt}

\selectlanguage{spanish}%
\appendix

\chapter{Documento de Lista de Riesgos\label{chap:Documento-de-Lista}}


\chapter{Documento Plan de Pruebas\label{chap:Plan-de-pruebas}}


\chapter{Documento Requerimiento del Software\label{chap:Requerimientos}}


\chapter{Documento Utilización del Componente para iOS\label{chap:Utilizaci=0000F3n-ios}}


\chapter{Documento Utilización del Componente para Android\label{chap:-Utilizaci=0000F3n-android}}
\end{document}
