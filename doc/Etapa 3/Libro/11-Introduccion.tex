
\chapter*{{\large Introducción}}

\selectlanguage{english}%
\addcontentsline{toc}{chapter}{Introducción}\foreignlanguage{spanish}{\thispagestyle{empty} }

\selectlanguage{spanish}%
El mundo de la tecnología cambia día a día, donde los usuarios experimentan
continuamente con una gran cantidad de formas de tecnologías creativas
e innovadoras. Cada semana salen al mercado aplicaciones móviles o
portales web que ofrecen servicios que hacen más fácil y entretenida
la vida de las personas, tales como las tiendas y bancos virtuales,
el correo electrónico y los mensajes instantáneos.

Es así que el consumo crece en igual medida, por lo que comunicar
en masa a través de este medio es de gran importancia para las empresas
que deseen dar a conocer sus productos o servicios. 

La colocación de publicidad en los dispositivos móviles es un suceso
reciente que ha dado buenos resultados. Mobile Media Networks se especializa
en esta área, siendo uno de sus servicios ofrecidos la distribución
de publicidad en aplicaciones de dispositivos móviles. Esta distribución
consiste en la colocación de la publicidad y en la presentación de
los reportes a los anunciantes. El reporte consta de datos importantes,
como impresiones, interacciones y el porcentaje de efectividad. Sin
embargo los métodos de mercado se hacen cada día más exigentes y ambiciosos
con la idea de promover acciones en los usuarios de la publicidad.
Los anunciantes desean dirigir su publicidad a quién muestre interés
por ella, a quien le sea relevante.

La empresa utiliza para la distribución de publicidad en los dispositivos
móviles, el componente AdField MobileX de Blackberry, en conjunto
con el sistema AdServer MobileX, basado en la plataforma OpenX. En
base a las necesidades se ha planteado la creación de los componentes
publicitarios para las plataformas iOS y Android, incluyendo nuevas
funcionalidades que el AdField no contempla actualmente.

El presente proyecto describe el proceso de desarrollo del componente
publicitario AdView MobileX, que permita la colocación de las creatividades
en las aplicaciones móviles, así como el reporte de las impresiones
y clics realizados, el envío de datos importantes de segmentación
al servidor, y el almacenamiento y rotación de las creatividades.
Fue desarrollado bajo la metodología AUP en un período de 20 semanas.

El informe desarrollado para este proyecto de pasantía larga presenta
una estructura en capítulos. El primer capítulo describe el entorno
laboral, para la ubicación del escenario del problema desarrollado.
El segundo capítulo comprende el marco teórico donde se exponen los
conceptos necesarios para la comprensión del problema y de la solución
realizada. El tercer capítulo muestra la metodología utilizada AUP.
El cuarto capítulo presenta el marco tecnológico describiendo las
herramientas utilizadas para el desarrollo del proyecto. En el capítulo
quinto se describe el desarrollo del proyecto y los resultados basado
en la metodología usada. Finalmente se plantean las conclusiones y
recomendaciones.

A continuación, se exponen los objetivos generales y específicos que
se busca alcanzar en este desarrollo, y el alcance que abarca este
proyecto. Esta información tiene la intención de que el lector comprenda
el contexto general del proyecto y conozca el propósito del mismo.

\vspace{7.5pt}


\noindent \textbf{Objetivo general}\vspace{7.5pt}


Diseñar un protocolo eficiente y un componente nativo para cada plataforma
(IOS y Android) que permita entregar y reportar los impactos publicitarios
asignados.\vspace{7.5pt}


\noindent \textbf{Objetivos específicos}\vspace{7.5pt}

\begin{itemize}
\item Diseñar un protocolo que permita enviar toda la información necesaria
para mostrar la publicidad y reportar de vuelta la correcta entrega
de los datos y las acciones tomadas por el usuario.
\item Diseñar un componente que se acople con el protocolo planteado y permita
la recepción y visualización de la información enviada por el AdServerMobileX.
\item Desarrollo de un componente nativo IOS a partir del diseño propuesto.
\item Desarrollo de un componente nativo Android a partir del diseño propuesto.
\end{itemize}
\vspace{7.5pt}


\noindent \textbf{Alcance}

\vspace{7.5pt}


El alcance de este proyecto es el desarrollo de librerías para la
introducción de publicidad interactiva en las aplicaciones nativas
con el objetivo de ofrecer una presentación personalizada. La entrega
del prototipo será 100\% funcional.

\selectlanguage{english}%
\pagebreak{}\selectlanguage{spanish}%

